\chapter*{Conclusion Part I} \label{chap:conclusionPart1}
\addcontentsline{toc}{chapter}{Conclusion Part I}

\textbf{conclusion MOSIM}
In this paper, we investigated the optimization of a carsharing \emph{one-way} system which can be integrated in a multimodal transportation system. Specifically, in dense urban context, the demand of those systems may originate from any other modes of transportation such as trains, subways, bikes, etc. The optimization focus on fleet dimensioning when station locations and accurate demand are given. We proposed an integer linear program based on flows and we suggested three optimization criteria: maximizing the overall demand the system can absorbed, and minimizing the number of vehicles and relocation operations needed. We also described how to extract the total number of vehicles from any optimal flow.

Two numerical experiments were made using an open-source solver (GLPK). The fist one investigated computation times over small instances generated by a random generator. It turns out that the latter is negligible compared to the model building time. Furthermore, we observed that for each non-integer optimal value we get solving the LP model, its integer part is always equal to the optimal value of the corresponding ILP model. This let us think that the LP solution must contain rounding errors and might be integer. We intend to clarify this point in further research. A graphical representation of a 3-dimensional Pareto frontier served to underline and to confirm that the optimizations criteria are in opposition.

The second experiment aims at specifying the time needed for building the mathematical model when the problem grows. As it is shown in the paper, the number of constraints of the mathematical model as well as the number of decision variables are both in the range of the number of arcs present in the time extended graph. We suggested to reduce this number in order to improve model generation time selecting only those at fixed time-step representing short distance relocation operations.

The choice of GLPK instead of a commercial solver is mainly motivated by our industrial context.
The good quality and the scalability of the solutions we obtained using this open-source solver allow us to transfer easily our methodology in an industrial environment.

~\\
\textbf{conclusion EWGT paper}
This paper presents an original mathematical model for carsharing system design purposes.
Our model is based on flows in a time expanded graph in order to group the vehicles passing through any road at any time. 
It is shown that vehicle routes can be recovered from any feasible flows.
The main theoretical consequence is that our optimization problem belongs to ${\cal NP}$.
An original polynomial sub-case, where all demands must be fulfilled has been also presented.

We shown experimentally that any random generated problem with realistic size ($50$ stations and $144$ time steps) can be exactly solved within a reasonable time.
We can withdraw that adopting a relocation strategy based on fixed time steps allow to handle larger problems, reduce graph density and largely improve solver calculation times while keeping good quality solutions.

The main conclusion is that our model can be consider to study vehicle relocation strategies in a real life context.
We also experimentally proved that decreasing the relocation frequency has a few impact on the total number of satisfied demands.
Scheduling operations every two hours for instance could help the operator organizing vehicle relocation routes while keeping a good level of service.
The next step will consist in finding more advanced and flexible relocation strategies and including other operational constraints such as a limited number of employees in charge of the vehicle relocation task.
