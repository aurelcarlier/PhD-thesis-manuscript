\chapter{The System Dimensioning Problem} \label{chap:sdp}
\epigraph{All human things are subject to decay, and when fate summons, Monarchs must obey}{\textit{Mac Flecknoe \\ John Dryden}}

\begin{bibunit}[ieeetr]
\minitoc
\vspace{2cm}

\begin{minipage}[c]{0.3\linewidth}
\includegraphics[width=\textwidth]{systemDimensionning}
\end{minipage}
\hfill
\begin{minipage}[c]{0.7\linewidth}
\begin{abstract}
This section aims at modelling our optimization problem using an integer linear program.
The inputs and the outputs of the problem are first described.
Second subsection is dedicated to the building of an oriented valued graph, namely the Timed Extended Graph. This graph, previously introduced by  Ahuja {\em et al.} \cite{ahujaNetwork1993}, allows to express all the constraints of the problem following the time and the space dimensions.
The third subsection introduces the decision variables of our optimization problem. More precisely, it is shown that vehicles can be equivalently aggregated into flows to express all the constraints and the criteria of our optimization problem.
Last subsection presents its formulation using integer linear programming.
\end{abstract}
\end{minipage}

\newpage
\section{Introduction : problem description}
The first part of this thesis is dedicated to the optimal design of a carsharing system when stations are fixed.
We assume that a time-dependant carsharing demand can be estimated from origins to destinations.
This demand covers a geographical area where a carsharing operator is interested to established its service.
Without any doubt, the demand estimation model as well as the time period considered in the study are critical inputs in the decision process.
This bias will be discussed afterwards.

In this work, the meaning of optimality will be considered as the will to capture the maximum number of demands.
Somehow, this objective is aligned with the maximization of the operator's revenues.
Indeed, the more demands the systems is able to fulfil, the more profits the operator can generate.

As seen in \cite{les_relocs_sont_importantes_et_necessaires}, the optimization process should also integrate vehicle relocation operations.
They are the only known "technique" to re-equilibrate the system and reach a higher satisfied demand.

In the reminder of this work, this system design problem will be refer as the \emph{System Dimensioning Problem} (SDP).
Its concise formulation could be :
\begin{quote}
``Considering potential one-way carsharing station locations and demands over time, what could be the optimal \emph{system configuration} capturing the higher number of demands ?''
\end{quote}

We define a system configuration as the quantification and/or the description of the following components:
\begin{enumerate}
\item the number of demands satisfied;
\item the number of vehicles needed to run the system;
\item the vehicle relocation operations.
\end{enumerate}

This chapter is organised as follow.
A first section is devoted to a brief state of the art and related models found in the literature.
Then, Section \ref{sec:mathModel} describes the mathematical materials and defines the problem mathematically.
Section 

% finir l'intro : organisation du chapitre

\newpage
\section{Related work}


\newpage
\section{Mathematical model} \label{sec:mathModel}
\subsection{Inputs and outputs of the problem}
% In order to describe the mathematical problem, let us define first the required inputs.
% As discussed later, those data can be extracted from various sources such as simulation tools or real operating carsharing system data for example. 

% time
As seen previously, the SDP is time-dependant.
A carsharing system evolves over time.
Vehicles are moving among stations, taking more or less time to join a destination according to their departure time.
As a consequence, a formal description of the time component is first needed.
In this work, we decided to consider a discrete set of time-steps, numbered from $1$ to $T \in \N$ and collected in a set called ${\cH}$.
For relevance issues, and especially for experiments purposes, this period has to cover a representative period of time, as an average weekday or an average week for instance.
Some discussions about it are given thereafter in the next chapter.
~\\

% stations and capacities
Carsharing stations are gathered in a set called ${\cN}$, and also numbered with natural integers from $1$ to $N > 1$.
Each station comes with $Z(i)$, the maximum size (\ie maximum capacity, in terms of number of parking places) of station $i \in {\cN}$.
The set of stations' capacities is denoted $\cZ$.
~\\

% demands
Then, the demand $D(i,j,t)$ contains the number of passengers wishing to join station $j \in \cN$ from station $i \in \cN$, when the departure time at $i$ is $t \in \cH$.
Note here that there is no condition on the station themselves, thus allowing the travel to be a ``round trip" or a ``one-way" travel.
In this latter case, $i$ is simply equal to $j$.
% All the demands are gathered in a set called $$.
~\\

% travel times
Finally, $\delta(i,j,t)$ represents the travel time it takes for a car-user from station $i \in \cN$ to station $j \in \cN$, when departure time from $i$ is $t \in \cH$.
We suppose that for any triple $(i,j,t)\in \cN \times \cN \times \cH$, $\delta(i,j,t)<T$.
This assumption comes from the fact that the highest distance from different stations is quite low (usually less than $150$ or $200$ kilometres) and that the time-steps are covering at least a day.

A feasible solution of our problem consists on a set of vehicle tours, each of them modelling the situation of a car at each time step.
The three criteria (the demand, the number of vehicles and the number of relocation operations) can be polynomially computed from any feasible solution.

\subsection{Time Extended Graph}

Including time in a network flow model can be done using Time Expanded Graphs (TEGs) as suggested by \cite{ahuja1993}.
Carsharing stations are duplicated at every discrete time-step so that links (arcs) between stations could represent time-dependent vehicle operations (staying parked, satisfying a demand or being relocated).

To deal with discrete-time dynamic networks, Ahuja {\em et al.} \cite{ahujaNetwork1993} suggested the use of \emph{time-space network}, also known as \emph{Time Extended Graphs} (TEG). 
It is  a static network constructed by expanding the original network in the time dimension, considering a separate copy of every node $i \in \cN$ at every discrete time-step $t \in \cH$. Thus, from the data listed above, let us  consider $G = ({\cX}, {\cA}, \cCap)$ as a TEG defined as follows:

\subsubsection{Set of nodes (MOSIM)}
Nodes are couples $(i,t)$ with $i \in \cN$ and $t \in \cH$ associated with station $i$ at time $t$. Formally, $\cX = \cN \times \cH$.

Let $\eta$ and $\theta$ the functions which reciprocally return the station and the time-step associated with every element of $\cX$. More formally,
$$\eta:  {\cX  \rightarrow  \cN} \mbox{ with }  x=(i,t) \mapsto \eta(x)=i$$
$$\theta: { \cX \rightarrow  \cH} \mbox{ with } x=(i,t) \mapsto \theta(x)=t$$

\subsubsection{Set of arcs (MOSIM)}
Edges are partitioned into $3$ sets ${\cA}_1$, ${\cA}_2$ and ${\cA}_3$ defined as follows:
\begin{itemize}
\item 
${\cA}_1$ is the set of arcs representing the possibility for the vehicles to stay at a station between two consecutive time-steps.
Formally, 
${\cA}_1 = \{(x,y) \in {\cX} \times {\cX} ~|~ \eta(x) = \eta(y) ~and~ \theta(y) = \theta(x) + 1 \mbox{ mod } T\}$.

\item 
${\cA}_2$ are the arcs associated with a demand: each demand $D(i,j,t)>0$ corresponds to an arc  $(x,y)$ with
$x = (i, t)$ and $y = (j, t + \delta(i,j,t))$. Set ${\cA}_2$ is then formally defined as
${\cA}_2 = \{(x,y) \in {\cX} \times {\cX} ~|~ D(\eta(x), \eta(y), \theta(x)) \not = 0 ~and~ \theta(x) + \delta(\eta(x), \eta(y), \theta(x)) = \theta(y)\mbox{ mod } T\}$.

\item 
${\cA}_3$ represents all the possible relocation operations over time.
Any triple $(i,j,t) \in \cN \times \cN \times \cH$ with $i \neq j$ is associated to an arc from $x = (i,t)$ to $y = (j, t + \delta(i,j,t) \mbox{ mod } T)$.
Thus, 
${\cA}_3 = \{(x,y) \in {\cX} \times {\cX} ~|~ \eta(x) \not= \eta(y) ~and~ \theta(x) + \delta(\eta(x), \eta(y), \theta(x)) = \theta(y)  \mbox{ mod } T\}$.
\end{itemize}
The total number of arcs is given by:
$$|{\cA}| = |{\cA}_1| + |{\cA}_2| + |{\cA}_3| = N \times T + M + (N \times T) \cdot (N - 1)$$
where $M$ is the number of requested demands. Since $M \ll N^2$, $|{\cA}| = \Theta(N^2\cdot T)$.

\subsection{Basic definitions (EWGT)}\label{subsec:basic}
The Time Expanded Graph is a valued directed graph ${\cal G}=({\cal X}, {\cal A}, u)$ such that
nodes represent stations states over the time period, \ie ${\cal X}={\cal N}\times {\cal H}$.
Any arc $a=(x,y)\in {\cal A}$ is associated to a possible move of a vehicle from node $x$ to $y$.
The capacity $u(a)$ is the maximum number of vehicles allowed on $a$.

Let the function $\eta:  {{\cal X}  \rightarrow  \cal N}$ with  $x=(i,t) \mapsto \eta(x)=i$ referring to 
the station of a node $x\in {\cal X}$. Similarly, 
$\theta: { {\cal X} \rightarrow  \cal H}$ with $x=(i,t) \mapsto \theta(x)=t$ is the step-time of $x$.
Let $\Gamma^-({\cal G}, x)$ and $\Gamma^+({\cal G},x)$ respectively denotes the set of immediate predecessors and successors of a node $x\in{\cal X}$ in ${\cal G}$, \ie
$\Gamma^-({\cal G}, x)=\{y\in{\cal X}~|~(y,x)\in{\cal A}\}$ and $\Gamma^+({\cal G},x)=\{y\in{\cal X}~|~(x,y)\in{\cal A}\}$.
We simply note $\Gamma^-({\cal G}, x)=\Gamma^-(x)$ and $\Gamma^+({\cal G}, x)=\Gamma^+(x)$ if no confusion is possible.

\subsection{Set of arcs (EWGT)}\label{subsec:arcs}
Arcs set ${\cal A}$ is partitioned into three sets ${\cal A}_1$, ${\cal A}_2$ and ${\cal A}_3$ defined as follows.
\begin{itemize}
\item
${\cal A}_1$ is a set of arcs  associated to vehicles staying in a same station between two consecutive time steps. Formally, 
\begin{equation}\label{eqn:defA1}
{\cal A}_1 = \{(x,y) \in {\cal X}^2 ~|~ \eta(x) = \eta(y) ~and~ \theta(y) = \theta(x) + 1 \mbox{ mod } T\}.
\end{equation}
The capacity of any arc $a=(x,y)\in {\cal A}_1$ with $i=\eta(x) = \eta(y)\in {\cal N}$ is $u(a)=Z(i)$.
\item
Any arc $a=(x,y) \in {\cal A}_2$ corresponds to a positive demand from $\eta(x)$ to $\eta(y)$ at time $\theta(x)$. 
The arrival time is $\theta(x)+\delta(\eta(x), \eta(y), \theta(x)) \mod  T$. Arcs set ${\cal A}_2$  is then formally defined as
\begin{equation}\label{eqn:defA2}
{\cal A}_2 = \{(x,y) \in {\cal X}^2 ~|~ D(\eta(x), \eta(y), \theta(x)) > 0 ~and~ \theta(x) + \delta(\eta(x), \eta(y), \theta(x)) = \theta(y)\mbox{ mod } T\}.
\end{equation}
The capacity of any arc $a=(x,y)\in {\cal A}_2$ equals  $u(a)=D(\eta(x), \eta(y),\theta(x))$.
\item
Elements from ${\cal A}_3$ model relocations.
Each arc $a=(x,y)\in {\cal A}_3$ is associated to a possible relocation from station
$\eta(x)$ to $\eta(y)$ at time $\theta(x)$:
\begin{equation}\label{eqn:defA3}
{\cal A}_3 = \{(x,y) \in {\cal X}^2 ~|~ \eta(x) \not= \eta(y) ~and~ \theta(x) + \delta(\eta(x), \eta(y), \theta(x)) = \theta(y) \mbox{ mod } T\}.
\end{equation}
The capacity $u(a)$ of any arc $a=(x,y)\in {\cal A}_3$ is not bounded.
\end{itemize}
The total number of arcs is then
$|{\cal A}| = \sum_{k=1}^3 |{\cal A}_k| = N\cdot T + M + (N\cdot T)\cdot (N-1)=N^2\cdot T+M$.
As $M \ll N^2$, $|{\cal A}| = \Theta(N^2\cdot T)$.
We observe that $|{\cal A}_3|\gg |{\cal A}_1 \cup{\cal A}_2|$ and that the number of arcs is proportional to ${\cal A}_3$.


\subsubsection{Arcs Capacity}
Associated with each arc $a=(x,y) \in {\cA}$, is given a capacity function $u: {{\cA}} \rightarrow \N$ corresponding to a 
maximum number of vehicles allowed on $a$. It is defined as follows for any arc $a\in \cA$:
\begin{numcases}
{u(a = (x,y)) =}
Z(\eta(x)) &  if $a \in {\cA}_1$ \nonumber \\
D(\eta(x), \eta(y),\theta(x)) &  if $a \in {\cA}_2$ \nonumber \\
+ \infty & if $a \in {\cA}_3$ \nonumber
\end{numcases}

For any arc $a=(x,y)\in {\cA}_1$, the maximum number of cars is the capacity of the station $\eta(x)=\eta(y)$.
It corresponds to the demand for any arc $a\in {\cA}_2$, and it is not bounded for relocation arcs $a\in {\cA}_3$.

% TEG example
%\begin{figure}[!t]
%\centering
%\includegraphics[scale=0.21]{tikx_arcsStock}
%\caption{Time-Extended-Graph example}
%\label{TEGExample}
%\end{figure}

%The figure \ref{TEGExample} aims to help represent each topology of each sets of arcs. Capacities over arcs of ${\cA}_1$ and ${\cA}_2$ are represented in brackets, whereas %those infinite of ${\cA}_3$ do not appear in the figure. Let's notice that this example don't represent all the arcs of the graph since it is very dense.\\

\subsubsection{Additional notations}
We denote by $\Gamma^-(x)$ and $\Gamma^+(x)$ respectively the set of immediate predecessors and successors of a node $x\in{\cal X}$, \ie
$$\Gamma^-(x)=\{y\in{\cX}~|~(y,x)\in{\cA}\}$$
$$\Gamma^+(x)=\{y\in{\cX}~|~(x,y)\in{\cA}\}$$

For any couple of time instants $\forall (t, t') \in {\cH}^2$, 
the number of time-steps between those two instants is defined through the following function 
$$\begin{array}{r c l}
\vartheta : {\cH}^2 & \rightarrow & \N \\
(t, t') & \mapsto & \vartheta (t, t')
\end{array}$$

\begin{numcases}{\text{with  } \vartheta(t, t') =}
t' - t & if $t \leq t'$ \nonumber \\
T + t' - t & otherwise. \nonumber
\end{numcases}

For each arc $a=(x,y)\in {\cA}$, let us define the boolean value $\epsilon_a$ as:
\begin{numcases} {\epsilon_a =}
0 &  if $\theta(x) \leq \theta(y)$ \nonumber \\
1 & otherwise \nonumber 
\end{numcases}
The time required for a movement from $x$ to $y$ is then given by 
the function 
$$\begin{array}{r c l}
\ell : {\cA} & \rightarrow & \N \\
a = (x, y) & \mapsto & \ell(a) = \theta(y)- \theta(x)+\epsilon_a \cdot T
\end{array}$$

By extension, if $\mu= (a_1, \cdots, a_p) \in {\cA}^p$ is a path of the TEG from $x$ to $y$, the value
$\ell(\mu)=\sum_{i=1}^p \ell(a_i)$ is the total time required for a vehicle going from $x$ to
$y$ following $\mu$.

For any time value $t\in \cH$, let us define the set ${\Coupe}_t(\mu)$ as
the arcs $a=(x,y)$ from $\mu$ starting at time $t$ or earlier but ending after
$t$.
Formally,
${\Coupe}_t(\mu)=\{a=(x,y)\in \mu~|~\vartheta(\theta(x),t)<\ell(a)\}$.


\subsection{Decision variables (MOSIM)}
%The aim of our modelling is to not consider if possible the vehicles directly to solve our optimization problem. The main reason is that this number may be quite important, and a vehicle tour may be associated to each of them.

%Consider vehicles directly in the model does not seem to be a good option. Although it would be possible to create boolean variables for each node of the TEG to track and follow the movement of vehicles over the day, their number may be too large.
%Thus, we first introduce flow variables over arcs in the Time Extended graph.
%It is shown then that feasible vehicle tours can be extracted easily from any feasible flow.

The aim of our study is to compute the planning of each vehicles during the period. 
At any time, each vehicle is either parked in a station or in transit between two stations. Its position over the period can be modelled as a vehicle tour \ie a circuit $c=(a_1,\cdots, a_p)$ in the TEG.

The size of any feasible solution may be highly reduced if we only consider the number of vehicles passing through each arc.  
%Each vehicle in the system is thus associated to a vehicle tour (even if the vehicle is not used, staying in a station).
%\subsubsection{Flow variables}
For each arc $a =(x,y) \in {\cA}$, we call $\varphi(a)$ the flow of vehicles transiting through the arc $a$. It can be interpreted as:
\begin{itemize}
\item the number of vehicle staying in station $\eta(x)$ between two consecutive time-steps $\theta(x)$ and $\theta(y)$, if $a \in {\cA}_1$;
\item the number of vehicle picked by users from station $\eta(x)$ at time $\theta(x)$ to station $\eta(y)$, if $a \in {\cA}_2$;
\item the number of vehicle relocated between stations $\eta(x)$ and $\eta(y)$ at time $\theta(x)$, if $a \in {\cA}_3$.
\end{itemize}

The total number of vehicles transiting to any node $x\in \cX$ is clearly constant, thus
$$\sum_{\substack{y\in \Gamma^-(x)}} \varphi((y,x))= \sum_{\substack{y\in \Gamma^+(x)}} \varphi((x,y)).$$

%\subsubsection{Building vehicles tours}

It is immediate that a feasible flow may be obtained from any feasible set of vehicle tours.
We prove in the following that vehicle tours may be easily computed from a feasible flow.

\subsection{Decision variables (EWGT)}
The aim of our study is to compute the planning of each vehicles during the period. 
At any time, each of them is either parked in a station or in transit between two stations. Its position over the period can be modelled as a vehicle tour \ie a circuit $c=(a_1,\cdots, a_p)$ in the TEG.

We show in the following that a feasible solution can be described by only considering the number of vehicles passing through each arc.
For each arc $a =(x,y) \in {\cal A}$, we call $\varphi(a)$ the flow of vehicles transiting through the arc $a$. 
It can be interpreted as the number of vehicle staying in station $\eta(x)$ between two consecutive time-steps $\theta(x)$ and $\theta(y)$ if $a \in {\cal A}_1$, or 
the number of vehicle moving from station $\eta(x)$ at time $\theta(x)$ to station $\eta(y)$ otherwise.

Since the total number of vehicles transiting to any node $x \in {\cal X}$ is constant, 
\begin{equation}\label{eqn:flowconservation}
\sum_{{y\in \Gamma^-(x)}} \varphi((y,x))= \sum_{{y\in \Gamma^+(x)}} \varphi((x,y)).
\end{equation}
A flow $\varphi:{\cal A}\mapsto\mathbb{N}$  is said to be feasible if $\forall a\in {\cal A}$, $\varphi(a)\leq u(a)$ and 
$\forall x \in {\cal X}$, the flow conservation equation (\ref{eqn:flowconservation}) is true.
A feasible flow may be easily obtained from any feasible set of vehicle tours.

We prove in the following that the reverse is also true, with the consequence that any feasible solution of our problem can be described using a flow.
Next lemma computes the exact number of vehicles associated to a constant unitary flow over a circuit $c$.
\begin{lemma}
Let $c$ be a circuit and $\varphi_c$ a feasible flow such that:
\begin{numcases} {\varphi_c(a) =}
1 &if $a$ belongs to $c$ \nonumber \\
0 &  otherwise. \nonumber 
\end{numcases}
The minimum number of vehicles to insure $\varphi_c$ is
$\frac{\ell(c)}{T}$.
\label{NbVoitCircuit}
\end{lemma}
\begin{proof}
For any time value $t\in \cal H$, let us define the set ${\cal {C}}_t(c)$ as
the arcs $a=(x,y)$ from $c$ starting at time $t$ or earlier but ending after
$t$.
Since $\vartheta(\theta(x),t)$ equals the number of time steps 
from $\theta(x)$ to  $t$, we get 
${\cal {C}}_t(c)=\{a=(x,y)\in c~|~\vartheta(\theta(x),t)<\ell(a)\}$.

Now, since $c$ is a circuit, the value $\vert {\cal C}_t (c) \vert$ is a constant $\forall t\in {\cal H}$
and corresponds to the total number of vehicles needed to insure a unitary flow over $c$. Let us prove
that $\vert {\cal C}_T (c) \vert=\sum\limits_{a\in c} \epsilon_a$. 
For that purpose, setting $B(c)=\{ a=(x,y)\in c ~|~\epsilon_a=1\}$, we show that $B(c)= {\cal C}_T (c)$.
\begin{itemize}
\item $B(c) \subseteq {\cal C}_T (c)$:
if $a=(x,y)\in B(c)$, then as $\theta(x)\leq T$, $\vartheta(\theta(x),T)= T-\theta(x)$.
Now, since $\epsilon_a=1$ and $\theta(y)\geq 1$, 
$\ell(a)=\theta(y)-\theta(x)+T \geq 1-\theta(x)+T >\vartheta(\theta(x),T)$ and $a\in {\cal C}_T (c)$.

\item ${\cal {C}}_T (c) \subseteq B(c)$:
let consider now an arc $a=(x,y)\in {\cal C}_T (c)$.
Since $\vartheta(\theta(x),T)= T-\theta(x) < \ell(a)$ we get that 
$\theta(y)-\theta(x)+\epsilon_a \cdot T > T-\theta(x)$ and thus 
$\theta(y)+\epsilon_a \cdot T > T$.  
As $\theta(y) \leq T$, we necessarily have $\epsilon_a=1$ and thus $a\in B(c)$.
\end{itemize}
Now, by Lemma \ref{timecircuit}, $\vert {\cal C}_T (c) \vert=\sum\limits_{a\in c} \epsilon_a=\frac{\ell(c)}{T}$, the lemma.
\end{proof}

\begin{theorem}\label{theo:decomp}
Any feasible solution $\varphi$ can be decomposed into a set of circuits ${\cal S}$ such that, for any
arc $a\in {\cal A}$, $\varphi(a)=\sum\limits_{c \in {\cal S}} \varphi_c(a)$.
\label{decomposition}
\end{theorem}

\begin{proof}
The proof is by recurrence on $n(\varphi)=\sum\limits_{a \in {\cal A}} \varphi(a)$.
The theorem is trivially true if $n(\varphi)=0$.

Let suppose now that $n(\varphi)>0$, thus there exists at least one arc $a=(x, y)\in{\cal A}$ with $\varphi(a)>0$. Set $\mu_0=(x, y)$ and let consider the sequence of paths $\mu_i$ built as follows:
\begin{enumerate}
\item
Stop the sequence as soon as  $\mu_i$ contains a circuit $c$;
\item
Otherwise, let $\tilde{a}=(\tilde{x},\tilde{y})$ the last arc of $\mu_i$. 
Since $\varphi(\tilde{a})>0$, 
the flow conservation equation (\ref{eqn:flowconservation}) insures that 
there exists an arc $a$ starting at $\tilde{y}$ with
$\varphi(a)>0$. We then set $\mu_{i+1}=\mu_i\cdot a$.
\end{enumerate}
As ${\cal G}$ has a finite number of nodes, the algorithm stops and a non empty circuit $c$ is  returned.
The flow $\hat{\varphi}$ defined as
\begin{numcases}
{\hat{\varphi}(a) =}
\varphi(a)-1 &  if $a \in c$  \nonumber \\
\varphi(a) &  otherwise. \nonumber 
\end{numcases}
is feasible with $n(\hat{\varphi})<n(\varphi)$, thus the theorem.
\end{proof}

Note that the  number of flow variables is a polynomial function on the size of the problem.
This is not true anymore for vehicle tours, which number can be exponential.
The consequence is that the determination of a flow is in ${\cal NP}$, which is not the case for the determination of vehicle tours.

\subsection{Solution and Objectives}
A feasible solution of our optimization problem is given by a set of vehicles, each of them associated with its position in the system
at any-time during the period studied.
The first objective is to maximize the total number of satisfied demands, \ie for which a vehicle is allocated. However, two other objectives
must be taken into account: each vehicle in the system is associated to a fixed cost, so that the total number of vehicles must be minimized. 
In the same way, vehicle relocations are fundamental for increasing the number of satisfied demands with a fixed  number of vehicles. However,
they cost an extra charge for the operator, and thus their number should also be limited.
In the following, the total number of vehicles and relocations are referred respectively by $C$ and $R$.

\subsection{Effective duration of a path or a circuit} \label{subsec:duree}
The duration of any path of a TEG may be easily evaluated. Indeed, 
for any couple of time instants $(t, t') \in {\cal H}^2$, 
let the function $\vartheta:{\cal H}^2 \mapsto \mathbb{N}^\star$
that computes the number of time-steps between those two instants. It is defined
formally as 
\begin{numcases}{\vartheta(t, t') =}
t' - t & if $t \leq t'$ \nonumber \\
T + t' - t & otherwise. \nonumber
\end{numcases}

For each arc $a=(x,y)\in {\cal A}$, let us define and set the boolean value $\epsilon_a$ to true if  $\theta(x)>\theta(y)$.
For any arc $a=(x,y)\in {\cal A}$, the effective time required for a move from  $x$ to $y$ is then equal to 
$\ell(a)=\theta(y)-\theta(x)+\epsilon_a \cdot T$.
By extension, if $\mu= (a_1, \cdots, a_p) \in {\cal A}^p$ is a path of the TEG from $x$ to $y$, the value
$\ell(\mu)=\sum_{i=1}^p \ell(a_i)$ is the total time required for a vehicle going from $x$ to
$y$ following $\mu$.
Next lemma evaluates the total time of any circuit $c$.
\begin{lemma} \label{timecircuit}
The total time of any circuit $c=(a_1,\cdots, a_p)$ is $\ell(c)=T\times \sum\limits_{i=1}^p \epsilon_{a_i}$.
\end{lemma}
\begin{proof}
Let $x_i$, $i\in \{1,\cdots, p+1\}$ be the sequence of elements from $\cal X$ such that, $x_{p+1}=x_1$ and
$\forall i\in\{1,\cdots, p\}$, $a_i=(x_i, x_{i+1})$.
The total time of $c$ is then
$$\ell(c)=\sum_{i=1}^p \ell(a_i) = \sum_{i=1}^p ( \theta(x_{i+1})- \theta(x_{i}) +\epsilon_{a_i} \cdot T) = T\times \sum_{i=1}^p \epsilon_{a_i},$$
the result.
\end{proof}

\subsection{Formal problem statements}
A formal definition of our main optimization problem, referred as  the basic carsharing problem with relocations [\textsc{bcpr}] can be stated as follows:

{\vspace{5pt}\noindent\textbf{Basic carsharing problem with relocations [\textsc{bcpr}]:}}
\begin{description}
\item[Inputs:] A set of stations ${\cal N}$ with their capacity $Z(i)$, $i\in {\cal N}$, time periods set ${\cal H}=\{1,\cdots,T\}$, 
travel times $\delta(i,j,t)$ for each triplet $(i,j,t)\in {\cal N}^2\times  {\cal H}$, a set of $M$ demands, fixed number of vehicles $C$ and relocation operations $R$.
\item[Question:]
What is the maximum number of demands $m\leq M$ that can be captured by a vehicle routing  of  at most $C$ vehicles and $R$ vehicle relocation 
operations during the considered period ${\cal H}$?
\end{description}
We will show that [\textsc{bcpr}] belongs to ${\cal NP}$ by modelling feasible solutions as a non classical flow problem. 

%%%
\subsection{How to recover the number of vehicles?}
Next lemmas characterize the total time of any circuit $c$ and the exact number of vehicles required for a unitary flow on $c$.

\begin{lemma} \label{timecircuit}
The total time of any circuit $c=(a_1,\cdots, a_p)$ is $\ell(c)=T\times \sum_{i=1}^p \epsilon_{a_i}$.
\end{lemma}
\begin{proof}
Let $x_i$, $i\in \{1,\cdots, p+1\}$ be the sequence of elements from $\cX$ such that, $x_{p+1}=x_1$ and
$\forall i\in\{1,\cdots, p\}$, $a_i=(x_i, x_{i+1})$.
The total time of $c$ is then
$$
\begin{array}{ll}
\ell(c) & =   \sum_{i=1}^p \ell(a_i)    \\
           & =   \sum_{i=1}^p ( \theta(x_{i+1})- \theta(x_{i}) +\epsilon_{a_i} \cdot T)\\
           & =   T\times \sum_{i=1}^p \epsilon_{a_i}. \\
\end{array}
$$
the result.
\end{proof}
\begin{lemma} \label{NbVoitCircuit}
Let $c$ be a circuit and $\varphi_c$ a feasible flow such that 
\begin{numcases} {\varphi_c(a) =}
1 &  if $a$ belongs to $c$\nonumber \\
0 & otherwise \nonumber 
\end{numcases}
The minimum number of vehicles to insure $\varphi_c$ is
$\frac{\ell(c)}{T}$.
\end{lemma}
%
\begin{proof}
The total number of vehicles needed at time $t\in {\cH}$ for $c$ is clearly $\vert {\Coupe}_t (c) \vert$.

We first show that $\vert {\Coupe}_T (c) \vert=\sum_{a\in c} \epsilon_a$. 
For that purpose, setting $B(c)=\{ a=(x,y)\in c ~|~\epsilon_a=1\}$, we prove that $B(c)= {\Coupe}_T (c)$.
\begin{itemize}
\item $B(c) \subseteq {\Coupe}_T (c)$:
If $a=(x,y)\in B(c)$, then as $\theta(x)\leq T$, $\vartheta(\theta(x),T)= T-\theta(x)$.
Now, since $\epsilon_a=1$, $\ell(a)=\theta(y)-\theta(x)+T \geq \vartheta(\theta(x),T)$ and
$a\in {\Coupe}_T (c)$.

\item ${\Coupe}_T (c) \subseteq B(c)$:
Let consider now an arc $a=(x,y)\in {\Coupe}_T (c)$. 
Since $\vartheta(\theta(x),T)= T-\theta(x)< \ell(a)$, we get
$\theta(y)+\epsilon_a \cdot T>T$.  As $\theta(y) \leq T$, we necessarily have $\epsilon_a=1$ and thus $a\in B(c)$.
\end{itemize}

Now, by Lemma \ref{timecircuit}, $\vert {\Coupe}_T (c) \vert=\sum_{a\in c} \epsilon_a=\frac{\ell(c)}{T}$. Lastly, the minimum number of vehicles to insure $\varphi_c$ is constant over $\cH$ and thus $\forall t\in \cH$,
the lemma.
\end{proof}

\begin{theorem} \label{decomposition}
Any feasible solution $\varphi$ can be decomposed into a set of circuits ${\cal S}$ such that, for any
arc $a\in G$, $\varphi(a)=\sum_{c \in {\cal S}} \varphi_c(a)$.
\end{theorem}
\begin{proof}
The proof is by recurrence on $n(\varphi)=\sum_{a \in {\cA}} \varphi(a)$.
The theorem is trivially true if $n(\varphi)=0$.

Let suppose now that $n(\varphi)>0$, thus there exists at least one arc $a=(x, y)$ with $\varphi(a)>0$. Set $\mu_0=(x, y)$ and let consider the sequence of paths $\mu_i$ built as follows:
\begin{enumerate}
\item
Stop the sequence as soon as  $\mu_i$ contains a circuit $c$;
\item
Otherwise, let $\tilde{a}=(\tilde{x},\tilde{y})$ the last arc of $\mu_i$. Since $\varphi(\tilde{a})>0$, following the
conservation low of $\varphi$ over nodes of $G$, there exists an arc $a$ starting at $\tilde{y}$ with
$\varphi(a)>0$. We then set $\mu_{i+1}=\mu_i\cdot a$.
\end{enumerate}
As $G$ has a finite number of nodes, the algorithm stops and a non empty circuit is then returned.
Let now define the flow $\hat{\varphi}$ as follows
\begin{numcases}
{\hat{\varphi}(a) =}
\varphi(a)-1&  if $a \in c$  \nonumber \\
\varphi(a) &  otherwise \nonumber 
\end{numcases}
$\hat{\varphi}$ is feasible with $n(\hat{\varphi})<n(\varphi)$, thus the theorem.
\end{proof}

Note that the  number of flow variables is a polynomial function on the size of the problem.
It is not the case for the vehicle tours, which number can be of exponential size.
The consequence is that the determination of a flow is in ${\cal NP}$, which is not the case for the determination of vehicle tours.

\subsection{Modelling of the optimization problem}
Three objectives are to be considered for our optimization problem: the main objective is to maximize the number of demands. The two other ones are to minimize both the number of relocations and the total number of vehicles.

As we shall see later, the number of demands and relocations are easily linearly expressed using the flows. Next theorem shows that it is also the case for the total number of vehicles:

\begin{theorem} \label{NbVoiture}
The minimum total number of cars required for a feasible flow $\varphi$ equals
$\sum_{a\in \cA} \varphi(a) \cdot \epsilon_a$.
\end{theorem}
\begin{proof}
Let ${\cal S}$ be a set of circuits obtained from the decomposition of $\varphi$
following Theorem \ref{decomposition} and let $V$ be the minimum number
of cars associated with $\varphi$.
By Lemmas  \ref{timecircuit} and \ref{NbVoitCircuit}, the total number of car of any circuit $c\in {\cal S}$ is
$$\sum_{a\in c} \epsilon_a = \sum_{a\in {\cA}} \epsilon_a \cdot \varphi_c(a)$$
and thus
$$V \leq \sum_{c\in {\cal S}} \sum_{a\in {\cA}} \epsilon_a \cdot \varphi_c(a).$$
Now, from  Theorem \ref{decomposition}, $\varphi (a)= \sum_{c\in {\cal S}}\varphi_c(a)$.
Thus, 
$$\sum_{c\in {\cal S}} \sum_{a\in {\cA}} \epsilon_a \cdot \varphi_c(a)=\sum_{a\in {\cA}} \epsilon_a \cdot \sum_{c\in {\cal S}}  \varphi_c(a)=\sum_{a\in {\cA}} \varphi(a) \cdot \epsilon_a.$$
Lastly, the total number of vehicles required at time $T$ to reach $\varphi$ is exactly $\sum_{a\in {\cA}} \varphi(a) \cdot \epsilon_a$, the theorem.
\end{proof}

The modelling of our optimization problem follows. $R$ and $C$ are fixed bounds for respectively  the total number of relocation operations and vehicles.
Equation $(1)$ is the maximization of the total demand.
Equation $(2)$ expresses the bound on the total number of relocation.
Equation $(3)$ expresses these on the total number of vehicles.
Equations $(4)$, $(5)$ and $(6)$ are lastly flow constraints.

{\noindent
\begin{minipage}{\linewidth}
\setlength{\mathindent}{0pt}
\begin{flalign} \label{objective_function_users}
\max ~~\sum_{\substack{a\in {\cal A}_2}} \varphi(a)
\end{flalign}
\begin{numcases}{s.t.}
\sum_{a \in {\cA}_3} \varphi(a) \leq R              &\\
\sum_{a\in \cA} \varphi(a) \cdot \epsilon_a \leq C  &\\
\varphi(a) \leq u(a)                                &$\forall a\in{\cal A}$\\
\sum_{\substack{y\in \Gamma^-(x)}} \varphi((y,x)) = \sum_{\substack{y\in \Gamma^+(x)}} \varphi((x,y)) &$\forall x\in{\cal X}$\\
\varphi(a) \in \N                                   &$\forall a\in{\cal A}$
\end{numcases}
\end{minipage}}
~\\The total number of equations is around $2|{\cA}| + N\times T = \Theta(N^2 \cdot T)$.

\newpage
\section{Theoretical result}
\subsection{Problem complexity}
intro.

\subsection{Polynomial sub-case}
This section aims to prove that the determination of a flow that satisfies all the demands 
without a constraint on the total number of relocations or vehicles is a polynomial problem.
The formal definition of this problem,  designated by \textsc{all-demands}, follows. 

\begin{pbDefinition} [\textsc{all-demands}]
\begin{description}
\item[Inputs :] A Time Expanded Graph ${\cal G}=({\cal X}, {\cal A}, u)$.
\item[Question :]
Is there a feasible flow $\varphi$ such that all the demands are fufilled, {\em i.e} $\forall a \in {\cal A}_2, \varphi(a)=u(a)$ ?
\end{description}
\end{pbDefinition}

Let $I$ be an instance of \textsc{all-demands}. We associate an instance of a max-flow problem $f(I)$ 
which network $\widehat{\cal G}=(\widehat{\cal X},\widehat{\cal A}, \widehat{w})$ is defined
as follows:
\begin{enumerate}
\item
Vertices are $\widehat{\cal X}={\cal X} \cup \{s^\star, t^\star\}\cup \{s_a, t_a, a\in {\cal A}_2\}$. $s^\star$ and $t^\star$ are respectively
the source and the sink of $\widehat{\cal G}$, while $s_a$ and $t_a$ are two additional vertices associated to any demand arc $a\in {\cal A}_2$.
\item
Arcs set is $\widehat{\cal A}={\cal A}_1 \cup {\cal A}_3 \cup \{ (x,t_a), (t_a,t^\star), (s^\star, s_a), (s_a,y), \forall a=(x,y)\in {\cal A}_2\}$.
\item
Maximum capacity of arcs are $\widehat{w}(a)=u(a)$ for $a\in {\cal A}_1 \cup {\cal A}_3$.
Otherwise, for any arc $a=(x,y)\in {\cal A}_2$, 
$\widehat{w}((x,t_a))=\widehat{w}((t_a,t^\star))=\widehat{w}((s^\star, s_a))=\widehat{w}( (s_a,y))=u(a)$.
\end{enumerate}
Note that this transformation is a polynomial function and does not depend on  the structure
of ${\cal G}$.

\begin{theorem}\label{theo:pol}
Let an instance of \textsc{all-demands} expressed by a TEG ${\cal G}$.
There exists a feasible flow fulfilling all the demands of ${\cal G}$ if and only if there exists a maximum flow in $\widehat{\cal G}$ of value $\sum_{a\in {\cal A}_2} u(a)$. 
\end{theorem}
\begin{proof}
Let suppose that $\varphi$ is a feasible flow of ${\cal G}$ that fulfils all the demands, \ie for any arc $a\in {\cal A}_2$, $\varphi(a)=u(a)$.
A flow $\widehat{\varphi}$ of $\widehat{\cal G}$ may be built as follows:
\begin{enumerate}
\item
$\forall a\in {\cal A}_1 \cup {\cal A}_3$, 
$\widehat{\varphi}(a)=\varphi(a)$;
\item
For any arc $a=(x,y)\in {\cal A}_2$, 
$\widehat{\varphi}((x,t_a))=\widehat{\varphi}((t_a,t^\star))=\widehat{\varphi}((s^\star, s_a))=\widehat{\varphi}( (s_a,y))=u(a)$.
\end{enumerate}
We prove that $\widehat\varphi$ is a feasible flow of $\widehat{\cal G}$ of value $\sum_{a\in {\cal A}_2} u(a)$. 
Indeed, let consider a node $x\in \widehat{\cal X}$.
\begin{enumerate}
\item Let suppose first that $x\in {\cal X}$. Then any demand arc $a=(y,x)\in {\cal A}_2$ ({\em resp.} $a=(x,y)\in {\cal A}_2$ ) of flow $\varphi(a)$ is associated in $\widehat{A}$ to an
arc $e=(s_a, x)$ ({\em resp.} $e=(x,t_a))$  with $\widehat{\varphi}(e)=\varphi(a)$. 
Thus, 
$$\sum_{a \in \Gamma^-({\widehat{\cal G}},x)} \widehat{\varphi}(a)=
\sum_{a \in \Gamma^-({\cal G},x)} \varphi(a)=\sum_{a \in \Gamma^+({\cal G},x)} \varphi(a)=
\sum_{a \in \Gamma^+({\widehat{\cal G}},x)} \widehat{\varphi}(a)\text{.}$$
\item
For any arc $a=(z,y)\in {\cal A}_2$, the two vertices $t_a$ and $s_a$ are such that
$$\sum_{e \in \Gamma^-({\widehat{\cal G}},t_a)} \widehat{\varphi}(e)=\widehat{\varphi}((x,t_a))=\widehat{\varphi}((t_a, t^\star))=\sum_{e \in \Gamma^+({\widehat{\cal G}}, t_a)} \widehat{\varphi}(e)
\mbox{ and }
 \sum_{e \in \Gamma^-({\widehat{\cal G}},s_a)} \widehat{\varphi}(e)=\widehat{\varphi}((s^\star,s_a))=\widehat{\varphi}((s_a, y))=\sum_{e \in \Gamma^+({\widehat{\cal G}},s_a)} \widehat{\varphi}(e)\text{.}$$
\item
Lastly, $$\sum_{e \in \Gamma^-({\widehat{\cal G}},t^\star)} \widehat{\varphi}(e)=\sum_{a\in {\cal A}_2} u(a)
\mbox{ and } \sum_{e \in \Gamma^+({\widehat{\cal G}},s^\star)} \widehat{\varphi}(e)=\sum_{a\in {\cal A}_2} u(a)\text{.}$$
\end{enumerate}
The consequence is that $\widehat\varphi$ is a feasible flow of $\widehat{\cal G}$ of value $\sum_{a\in {\cal A}_2} u(a)$. 

Conversely, any feasible flow of $\widehat{\cal G}$ of value $\sum_{a\in {\cal A}_2} u(a)$ verifies
that, for any arc $a=(x,y)\in {\cal A}_2$, 
$\widehat{\varphi}((x,t_a))=\widehat{\varphi}((t_a,t^\star))=\widehat{\varphi}((s^\star, s_a))=\widehat{\varphi}( (s_a,y))=u(a)$.
A feasible flow for ${\cal G}$ can be easily obtained by setting:
\begin{enumerate}
\item
$\forall a\in {\cal A}_1 \cup {\cal A}_3$, 
${\varphi}(a)=\widehat{\varphi}(a)$;
\item
For any arc $a=(x,y)\in {\cal A}_2$, $\varphi(a)=u(a)$,
\end{enumerate}
the theorem.
\end{proof}

According to \cite{ahuja1993}, the existence of a maximum-flow of a fixed value is a polynomial problem. The following corollary is thus a consequence
of Theorem \ref{theo:pol}:
\begin{corollary}
\textsc{all-demands} is polynomial.
\end{corollary}

\newpage
\addcontentsline{toc}{section}{Bibliography of chapter \thechapter}
\renewcommand{\bibname}{Bibliography of chapter \thechapter}
\putbib[bib/biblio]
\end{bibunit}
