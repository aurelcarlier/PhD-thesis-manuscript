\chapter{Background and Problem description} \label{chap:backAndPb}
\begin{bibunit}[ieeetr]
\minitoc
\vspace{2cm}

\begin{minipage}[c]{0.45\linewidth}
\includegraphics[width=\textwidth]{cube}
\end{minipage}
\hfill
\begin{minipage}[c]{0.45\linewidth}
\begin{abstract}
blabla\\
blabla\\
blabla\\
blabla\\
blabla\\
\end{abstract}
\end{minipage}

\newpage
\section{Carsharing systems : presentation, urban impacts and challenges}
$>$ Why sharing is a good solution ?

One of them is carsharing system which involves a small to medium fleet of vehicles, available at several stations, to be used by a relatively large group of members \cite{shaheen_short_1999}.
Although the first experiment of carsharing is identified in \cite{shaheen_short_1999} as Sefage, a Swiss company created in 1948, it became popular worldwide since the early 1990's.
Today, the actual largest carsharing company in the world is Zipcar, which was founded in 2000, and has a fleet of $10 000$ vehicles and $850 000$ members mainly in the US. % and a link to the zipcar web page
This service is a real alternative to private car and release the user from constraints related to individual property, since the carsharing company is in charge of insurance, maintenance, fuel (or electricity), taxes, depreciation, etc.
Different studies (see for example \cite{litman_evaluating_2000, prettenthaler_ownership_1999}) have evaluated that for a user driving less than $10 000$ kilometres per year (as much as $15 000$ km/y), it also could be a real alternative to private car, in a financial way, depending on local costs.

\subsection{Types of carsharing systems}

\subsubsection{Round-Trip systems}
Nowadays, we can find over the world two different types of carsharing systems.
Historically, the first one requires users to return vehicles to the station they were picked up.
These are called ``round-trip'' carsharing systems and are the most common.
They are simple to design and manage since the incoming demand in each station is sufficient to plan stocks.
The user behaviour in such systems is mainly oriented to leisure and household shopping purpose \cite{barth_shared_use_2002, costain_synopsis_2012}.

\subsubsection{One-way systems}
The second one, called ``one-way'' carsharing system, is much more flexible since it allows users to pick up a vehicle from a station and return it in a different one, which can be different from the origin.
Unfortunately, this greater flexibility comes with hard operational problems, listed and described below, due to the uneven nature of the trip pattern in urban areas.
However, let notice that despite these difficulties for the operator, one-way system captures more trips than the alternative system thanks to this flexibility which is, as showed in \cite{efthymiou_which_2012}, a critical factor to join a carsharing scheme.

\subsubsection{Free-floating systems}



\subsection{Impacts on the transportation system}
\subsubsection{Urban mobility}

In the last decade, several authors have showed that these systems have a positive impact on urban mobility, mainly because of higher utilization rate than private vehicle \cite{litman_evaluating_2000, schuster_assessing_2005}.
Indeed, shared vehicles spend more time on the road and less time parked (which represent for a private car almost 95\% of its total use time, as mentioned in \cite{transflash_2013}), thereby decreasing parking requirements in dense areas \cite{mitchell_reinventing_2010} and reducing the average number of vehicles per household \cite{martin_impact_2010, ter_schure_cumulative_2012}.
It also decreases the total number of vehicles on the road, since one vehicle can be driven by several users and thus improving the traffic fluidity.

\subsubsection{Environmental effects}
Furthermore, it is now recognized that carsharing systems have positive environmental effects.
It reduces greenhouse gas (GHG) and CO2 emissions \cite{martin_greenhouse_2011, firnkorn_what_2011} and provides noise reduction since electric cars are quitter than thermal ones.
In addition, the reduction of parking demand can be used to reallocate the land for additional green spaces, new mixed-use development, or other community needs \cite{cohen_carsharing_2008}.


%As we explain below, these results are very important to calibrate simulation tools and have a good estimation of the demand.

%Because of their greater scientific interest compared to the ``round-trip'' carsharing system, we will now focus on the ``one-way'' carsharing system and develop more explicitly the main research areas and their main outcomes.

\subsection{The carsharing demand and related problems}

\subsubsection{The demand}
Some studies are also conducted to characterize and analyse who the users of these systems are.
Using statistical data and surveys, most studies demonstrated important tendencies: high correlation with the use of public transports, dwelling place in dense areas \cite{cervero_city_2003, millard_ball_car_sharing_2005, burkhardt_who_2006}, age between mid-30s to mid-40s, people highly educated and environmentally aware \cite{costain_synopsis_2012, efthymiou_which_2012, millard_ball_car_sharing_2005, brook_carsharingstart_2004, lane_phillycarshare_2005, zheng_carsharing_2009}.
Moreover, as showed in \cite{costain_synopsis_2012, efthymiou_which_2012, zheng_carsharing_2009}, the accessibility to the stations, in terms of distance between home/work and the nearest station, is a critical factor to joining a carsharing system.

To be able to give a good prediction of a carsharing service, it's really important to identify the main factors which generate and influence the demand.
Stillwater et al. \cite{stillwater_carsharing_2009} have concluded that the most significant variables were: the street width (-), the provision of a railway service (-), the percentage of drive-alone commuters (-), the percentage of households with one vehicle (+), and the average age of the stations (+).
We indicate by (-) or (+) when the indicator is negatively or positively related to carsharing demand.
The street width and the percentage of drive-alone commuters may not have a clear intuitive explanation at first sight although those metrics are significantly related to the level of carsharing demand.
The authors postulate that street width contains informations about pedestrian environment (where narrow streets are more pedestrian friendly) and about the land use in general (narrow streets trend to denote older residential or mixed-use development) witch make sense since carsharing and walking behaviour are known (see e.g. \cite{cervero_city_2003}) to be strongly related.
The proportion of drive-alone commuters are negatively related because these people generally would already own vehicles and high level of vehicle commuting tend to signify a neighbourhood that has poor public transit or other high-density mode amenities.

Another study conducted by Ciari et al. \cite{ciari_estimation_2013} uses an activity-based micro-simulation model to estimate travel demand and understand the effect carsharing system on urban mobility, considering others transportation modes such as public transport, car, bicycle and walking.
They suggested and evaluated a cost function in an open-source activity-based multi-agent simulator called MATSim \cite{matsim_webPage} which informs the user of the cost using carsharing as mode of transportation.
Thus, they have led to a modal split model, giving plausible results compared to real data (the urban area of Zurich, Switzerland), which can capture the proportion of total demand that could use this mode of transportation, depending on the access to the cars and the dependent fee structure.
In most cases, studies are context specific.
Trip patterns and travel behaviour can be different from one country to another since it is related to local and regional characteristics (culture, habits, etc.), making the standardization more complex.
Furthermore, as mentioned in \cite{jorge_carsharing_2013}, demand estimation has not so far been addressed in the literature for one-way carsharing systems, and a relevant model for such models is nowadays not available.
This is a real challenge for the future since it's reasonable to think that one-way carsharing systems will be increasingly present in the coming years.

\subsubsection{The vehicle imbalance problem}

As said before, the most challenged carsharing system is the one-way system. Since the arrival station is not necessary the same than the departure station in those systems, it induces and generates imbalance issues.
Thereby, a lot of efforts are made to understand the dynamics and find possible solutions to handle it.
The intuitive approach for solving the vehicle imbalance problem is to consider that the operator have to do periodic relocation operations among stations.
Some studies, using discrete event simulation models (see for example \cite{barth_simulation_1999, kek_relocation_2006, kek_decision_2009}), help operators to manage their systems minimizing available resources (such as vehicles and staff members), while maintaining certain levels of service.
The model presented in \cite{kek_decision_2009} has been tested and validated using real data (a one-way carsharing system called Honda ICVS) and proposed solutions reducing staff cost of about 50\%.
Other authors have explored the problem under the optimization methods perspective.
For instance, the model proposed in \cite{nair_fleet_2011} is a stochastic mixed-integer programming (MIP) model with the objective of generating least-cost vehicle redistribution where the demand is known probabilistically.
In \cite{smith_rebalancing_2013}, the authors find the optimal rebalancing strategy solving two different linear programs in a fluid model of the system : one in order to minimize the number of rebalancing vehicles, the other for minimizing the number of rebalancing drivers (staff members), considering that the number of waiting customers remains bounded.
The authors state that the ``two objectives are aligned'' and concluded that, for Euclidean network topologies, the numbers of drivers needed is between 1/4 and 1/3 of the number of vehicles.
Another innovative approach is to consider that clients can be used to relocate the vehicles through various incentive mechanisms.
Prices could be used in order to encourage users to sign up to ``trip splitting'' and ``trip joining'', as showed in \cite{barth_user_based_2004}.
The principle is very simple: when users wanted to travel from a station with shortage of vehicles to another one with an excess they were prompted to share the ride in a single vehicle (trip joining), while, conversely, when they wanted to travel from a station with too many vehicles to one with a shortage they were encouraged to drive separate vehicles (trip splitting).
But despite the fact this strategy effectively balances the system in theory, it relies on assumptions that may be unrealistic in practice.
For instance, it's not relevant if a majority of travellers value privacy and convenience over minor cost saving, or if trip-joining policies make carsharing similar to carpooling (which has severe sociological barriers associated with riding with strangers, mainly for safety and security reasons as said in \cite{chan_ridesharing_2012, correia_carpooling_2011}), or finally, with respect to trip splitting, if users simply do not want to be divided.

\subsection{Outlooks}

For an interesting and complete literature review of the carsharing systems and their attached problems, please refer to \cite{jorge_carsharing_2013}.
This paper has mostly inspired and directed this writing.
In this brief overview, we have seen that carsharing systems can be a real alternative to private vehicles.
They have positive impacts on urban mobility, positive environment effects, and thus help to provide solutions for urban transportation problems.
For instance, congestion, pollution and parking demand can be greatly enhanced with such systems.
We also saw that a lot of studies try to figure out who the users of those systems are, primarily
using regression and cluster analysis.
This social characterization is essential since it is directly related to demand modelling research, which attempts to estimate the proportion of
travellers that could join and use a carsharing system.
Unfortunately, the articles reviewed on demand estimation have not yet taken into account the one-way carsharing systems.
Since they will likely spread out in the near future, it will be crucial to develop realistic models that can handle those models, integrating carsharing systems in a multimodal environment and measure their impact on the global transportation scheme.
And last but not least, we presented two major problems inherent to one-way carsharing systems.
Let's notice and emphasize that accurate outputs of the demand estimation is a paramount data for both problems.
The first problem is the vehicle imbalance problem that involves vehicle repositioning by the operator or by the user in order to provide the best operational configuration.

Simulation and optimization are the two most commonly used techniques for dealing with this problem.
Research studies concluded that the ability to balance vehicles between stations is crucial for the overall system performance.
More precisely, good relocation strategies allow the system operates at a reliability level that could not be achieve otherwise.
As we seen before, some of them helped dividing staff cost by two \cite{kek_decision_2009}.
The second problem is the station location and  fleet dimensioning problem.
For now, very few studies have been conducted.
They all use mathematical integer programming (MIP) models and thus are limited because of the hard complexity of this problem.
The large number of variables needed to integrate several decisions into the same problem in a real case study context is indeed a real challenge in this field.
Actually, this last problem represents the core of this thesis and the scope of our research.
Therefore, the next chapter will introduce more precisely the reasons why we were especially interested by those aspects and give a mathematical formulation of the problem we are working on.


\newpage
\section{Demand estimation}
\subsection{Assumptions}
\subsection{MIC integration}
\subsection{Random data generator}


\newpage
\section{Problem description}


\newpage
\addcontentsline{toc}{section}{Bibliography of chapter \thechapter}
\renewcommand{\bibname}{Bibliography of chapter \thechapter}
\putbib[bib/biblio]
\end{bibunit}
