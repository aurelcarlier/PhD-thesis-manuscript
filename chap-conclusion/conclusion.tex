\chapter{Conclusion} \label{chap:conclusion}
%\begin{bibunit}[ieeetr]
\minitoc
\vspace{4cm}
\begin{minipage}[c]{0.35\linewidth}
\includegraphics[width=\textwidth]{conclusion}
\end{minipage}
\hfill
\begin{minipage}[c]{0.6\linewidth}
\begin{abstract}
This final Chapter concludes the thesis.
After recalling the addressed problems and the context of this research, main results and contributions are summarized.
Some perspectives and opened problems are finally provided.
\end{abstract}
\end{minipage}



\newpage
\section{Addressed problems}
The popularity of station-based carsharing systems has recently led to an increasing interest in behavioural user analysis and system management.
Especially for the one-way version of the service where users are allowed to pick-up and return a vehicle at any dedicated station, the research community intensified its attention on optimal system design and optimal vehicle relocation strategies.
Indeed, the uneven nature of the demand for mobility in urban areas unbalances the system bringing many operational challenges.


\medskip
In this thesis, we investigated two problems arising in the design of station-based one-way carsharing systems.
The first one is dedicated to the optimal system dimensioning, referred in this manuscript as the System Dimensioning Problem {\SDP}.
For fixed and known station locations, it aims at determine global indicators, such as the minimum number of vehicles and the optimal station capacities (number of parking places), allowing the system to run at its highest potential.
Throughout this report, we considered the system efficiency (or the system potential) as the number of carsharing demands vehicles can satisfied along a specific time period.
In practice, we based our experimentations on data randomly generated or estimated during a typical weekday.%, where demand profiles are easily .

The second problem deals with optimal station locations and is referred as the Station Location Problem {\SLP}.
A key factor in increasing the performance of carsharing system is the ability to make the service accessible at relevant places.
As a generalisation of the {\SDP}, this problem focuses on identifying the right sub-set of stations among a set of potential carsharing sites that captures the highest number of demands.


\medskip
Besides determining optimal system components related to the system design (\eg the optimal fleet size, the station locations or the stations' capacities), many studies underlined that the approach must include important mechanics inherent to one-way carsharing systems.
For instance, the necessity for the carsharing operator to balance its vehicle fleet during the day is absolutely necessary to achieve a good service quality.
Those operations imply having its own resources (operator's employees also known as jockeys) to move vehicles among stations, generally from ones where there is an exceeding of vehicles to one in deficit.
As a consequence, additional constraints coping with the total number of jockeys and total number of relocation operations were also included.


\medskip
Today, carsharing operators are increasingly incorporating electric vehicles in their vehicle fleet. 
Despite the fact that such vehicles reinforced the positive impacts of carsharing on the environment, they also bring out some technological and practical challenges.
More particularly, the battery range and the power supply at charging points may constitute significant factors to be considered in the global system design process including electric cars.
The relatively limited autonomy of currently available electric cars constraint the vehicles to stay in station to be recharged.
The charging time depending on the power supply granted in stations, recharging a vehicle can make it unavailable for a long time ($6$ hours to fully recharge a $22$ kWh battery).


\bigskip
We recall in this chapter the main results obtained in this thesis with respect to the above mentioned problems.
Both theoretical and industrial contributions are summarized.
Some perspectives and future work are finally discussed.


\newpage
\section{Results and contributions}
\subsection{Mathematical models for carsharing system design}
Linear programming is an intuitive approach often used in the literature to deal with carsharing system design.
Whether to evaluate the system performance, sizing different components such as the number of vehicles or even to find accurate station locations, many studies report on a system modelling based on Time Expanded Graphs (TEG).
Indeed, since carsharing systems are strongly time-dependant (users' requests, travel times, electric recharge, vehicle relocation operations, etc.), models may account somehow for their temporal evolution during the day.

Carsharing TEGs are directed weighted graphs where nodes and arcs stand respectively for carsharing stations and vehicle operations.
Nodes are duplicated over a set of discrete time-steps, expanding the original network in the time dimension.
Arcs describe temporal linkage between nodes (carsharing stations) and are classified into three categories according to the nature of the operation.
The list includes stock arcs (when vehicles are parked in station), demand arcs (when vehicles are used to satisfy a request) and relocation arcs (when vehicles are relocated by a jockey).
Considering those operations led to the fact that at any time, each vehicle is either parked in a station or in transit between two stations.
Arc weights denote maximal station capacities (\ie maximum number of parking places) on stock arcs and number of requests on demand arcs.
Finally, note that considering a representative period of time makes the graph highly cyclical.
Indeed, the resulting arrival time of any operation is calculated modulo the global time period.
In practice, when an operation starts near the end of the day, it terminates early in the morning.
Required inputs to build such TEG are temporal sets of carsharing demands, travel times between stations and vehicles relocation strategies.


\medskip
Based on the TEGs representation of carsharing systems, Chapter \ref{chap:sdp} presented an Integer Linear Programming model (ILP) dealing with the {\SDP}.
The model considers integer flow variables over arcs accounting for vehicle flows moving in the system.
The objective is to find a flow passing through the maximum number of demand arcs.
Classical flow constraints (flow conservation on nodes and flow capacities on arcs) ensure the flow feasibility.
Additional constraints limits the total number of vehicles relocation operations, jockeys and vehicles.

The optimal system dimensioning can then be deduced from any feasible solution.
The resulting flow can be interpreted as vehicle routes, \ie vehicle itineraries, since every unitary cycle flow represents a single vehicle moving though the system (\see Theorem \ref{th:decomp}, page \pageref{th:decomp}).
Looking at the maximum flow transiting in a station over time allow to deduce the minimum number of required parking places in this particular site.
Finally, the total number of vehicles can be recovered from the flows passing though temporal cuts (\see Theorem \ref{NbVoiture}, page \pageref{NbVoiture}).

\medskip
Dealing with the {\SLP} in Chapter \ref{chap:slp} led to enhanced the ILP model with additional variables.
For this problem, instead of carsharing stations, nodes in the TEG stand for potential carsharing sites.
Decisional boolean variables associated with each site indicate its operational status (selected or not selected).
Dedicated constraints control the inner and outer flow at nodes so that vehicles can park or pass through stations if and only if the corresponding site is selected.
Finally adding a specific constraint under the maximum number of opened sites complete the linear model. 
The latter produces the same outputs that the previous one and  provides additional information about the opened sub-set of stations allowing the system to capture the highest number of demands.

\medskip
We have seen in a dedicated part that including energy components was not possible with the current TEG.
Because flows are not unitary, the interpretation of an optimal solution as a set of unitary cycles can lead to multiple possibilities.
Besides, since energetic components are not considered in the ILP, there is no guaranty that the interpretation asserts the induced constraints.
As a consequence, we presented in Section (\ref{sec:graphTransformations}) a graph transformation supporting unitary flows.
The ILP model is also enhanced with an additional group of decisional variables, referred as \emph{flow affectation} variables, which report the direction (\ie the upcoming arc) taken by the flow at each node of the TEG.
We show that their number can be significantly reduced identifying symmetrical situations.
Finally, additional costs on arcs account for the energy consumption it takes to perform the related operation.
For demand or vehicle relocation arcs, the consumption is the travel distance (penalized during traffic peaks) whereas for stock arcs the negative value (negatived cost) represents the number of kilometres the battery can be recharged during one time-step.
For each arcs, a dedicated variable tracks the remaining battery level of the flow passing by before the operation.

% From a theoretical perspective

\subsection{Experimental observations}

According to the size of the problem, running ILP models to achieve good system configurations can take quite a long time.
In our flow-based approach, the number of variables is related to the TEG density, \ie the total number of arcs.
Controlling it is fundamental to achieve real size and relevant instances of the problem.

\medskip
We have seen in Chapter \ref{chap:sdp} that this density depends on three parameters: the number of stations, the number of time-steps, and the vehicle relocation strategy.
Among these three factors, the number of vehicle relocation arcs plays a major role.
Considering possible relocation operations at any time lead to generate TEGs where relocation arcs represent almost $98$\% of the global graph density.
Experimental results show that a good level of the demand satisfaction can be still achieve applying vehicle relocation operations at wider time intervals, \eg every hours.
Those relocation scenarios allow to drastically reduce both graph densities and computation times while keeping a good level of service.

\medskip
Introducing energy components was a tough challenge.
In Chapter \ref{chap:slpEnergyExp} we evaluated the {\ENERGY} MIP model on a realistic scenario based on $15$ stations and $30$ vehicles.
Assuming electric cars with an autonomy range of $160$ kilometres and charging points power supply of $3,7$ kW, we observed that a half battery capacity is sufficient to achieve the highest demand satisfaction level.
This residual capacity opens many perspectives and we suggested many practical benefits.
For car manufacturers interesting in vehicle design for carsharing usage, it may represents significant financial savings since battery' prices still represent an important part of the global electric vehicle price.
Also, the highlighted overcapacity of electric batteries brings out other advantages more oriented to the efficiency and the optimization of the whole system.
Practical applications include better battery lifespan, additional energy supply to improve consumption peak curtailments and a better flexibility at managing the energy for the system itself.


\subsection{Random data generator}
Dealing with the lack of data is currently a critical issue. %with respect to carsharing 
On the one hand, today's operating systems do not shared their data or \emph{open} them to the community.
At this time, only sporadic and limited data is available, even for research purposes.
In addition, reported data (often press releases) concern the physical description of the system (number and location of stations, number of vehicles, etc.) or global statistics about operational indicators, such as the average number of requests a vehicle satisfies during a day and its average time spent on road.
Neither the users daily requests, nor the detailed vehicles relocation operations are today available.

On the other hand, the actual research on demand modelling and demand estimation is not enough accurate to predict and anticipate the observed patterns in urban contexts.
Numerous incentives at determining the modal part of carsharing among the global transportation demand have been carried out but in practice, their usage admits some hard limitations.
Often context specific or neglecting the structural configuration of the systems (which is known to slightly influence the demand), advanced studies have not yet succeed to come up with models that accurately estimate a real carsharing demand.

\medskip
In order to test and evaluate the different mathematical models, we developed and released an open-source software designed to randomly generate data for one-way carsharing systems \cite{csgen}.
The tool positions a set of carsharing stations over a configurable territory, and generates temporal demands over the system with respect to a global demand profile.
Over the day, the mobility pattern emulates a centroid configuration where most of the morning demands are oriented to the center of the city, whereas the opposite phenomenon is observed in the evening.
The generator also produces travel times based on inter-station distances and an average vehicle speed.
Note that they are penalized for settable pick hours (morning and evening) in order to handle and reproduce travel conditions during traffic congestion.
More details are given in Chapter \ref{chap:backAndPb}, page \pageref{csgeneratorDescription}.


\subsection{Industrial decision support tool}

\begin{figure}[t]
\includegraphics[width=\textwidth]{donnees}
\caption{Graphical visualisation of carsharing data over the city of Paris.}
\label{fig:donnees}
\end{figure}

\begin{figure}[t]
\includegraphics[width=\textwidth]{resultats}
\caption{Graphical visualisation of an optimal solution.}
\label{fig:resultats}
\end{figure}

As an active actor of the numerical transformation, the IRT SystemX develops and deploys innovative services dedicated to both academic and industrial actors.
In the domain of transportation and mobility, industrial challenges includes the conception of suitable transportation models at different scales, the design of modular and re-usable architectures and the development of platforms supporting services for stakeholders.
In this context, a dedicated platform called MOST provides the IT environment for model developments, simulation, supervision and optimization tools for the study of smart territories.

\medskip
The mathematical linear models presented along this thesis have been implemented and integrated into this platform.
As a Java Application Programming Interface (API), the tool provides functionalities and routines to build and resolve any instance of carsharing problems addressed in this manuscript.
Dedicated methods allow to specify and set various problem parameters (as inputs of the problem) including upper bounds for the number of vehicles, jockeys, station capacities, vehicle relocation operations, station locations as well as carsharing demands, temporal set definition and relocation strategies.
The tool also offers the possibility to produce realistic data using the random generator for one-way carsharing demand presented previously.
In addition, its integration within a multi-modal demand estimation software allow to recover real data processed from mobility surveys.
Although still under development, this specific part of the platform aims to extract the timestamped carsharing demand (O/D temporal matrices) from macroscopic transportation flows, providing valuable data for the carsharing application.

\medskip
In practice, the carsharing API is supported by graphical interfaces which facilitate the program usage.
Temporal data can be apprehended and examined through coloured maps, charts and global indicators reflecting the dynamics of the system.
Outputs and results of an optimization can also be visualized in detail.
Especially the vehicles itineraries and the stations' energetic consumption profile can be displayed over time.
Figures \ref{fig:donnees} and \ref{fig:resultats} illustrate some graphical frames provided by the tool.
Both depicted a specific carsharing use-case on the city of Paris.
Figure \ref{fig:donnees} represents the global demand over the different districts of the city whereas Figure \ref{fig:resultats} shows the results of an optimization with the proportion of satisfied demands over the territory.



\section{Perspectives}
% limitations
% ameliorations > better demand estimation
In this thesis, we addressed strategical aspects of carsharing system design problem.
To come up with relevant results, the model must be set up with realistic data and deals with real size instances.
Unfortunately, as said before, neither real carsharing data nor accurate estimation demand models for carsharing demand that are not context-specific and applicable for one-way travels are yet available.
Hence, we believe that significant efforts in this direction may result in more valuable and accurate analysis.
The random data generator as well as the software integration of the carsharing module into the estimation demand tool has been achieve in that sense and several studies are currently ongoing.


\medskip
% problem complexity
From a theoretical point of view, the {\SDP} problem complexity remains unknown.
Despite the fact that a polynomial sub-case has been exhibited in Chapter \ref{chap:sdp} and that the problem presents many similarities with a max-flow formulation, its exact definition does not fit with standard flow problems description.
The closer formulation seems to be the one of the circulation problem where there is no sources, no sinks and attempt to find a feasible flow in the graph \cite{ahuja_network_1993}.
As far as we know, the objective to maximize the flow passing through a specific subset of arcs (in our case the set of carsharing demand arcs) does not exist in the literature.


\medskip
% ameliorate the solver calculation times > heuristic
With respect to the mathematical approach using linear programming, some limitations about solver capabilities have been observed and experienced in practice.
Clearly, solving times depend on the available computing power and the size of the carsharing system you chose to design.
The inclusion of the energy for instance requires much more memory (the TEG is larger) and generates higher solving times since the resulting problem induced more decisions.
Basically, the {\SDP} model can solve instances with $100$ stations and $5$ minutes times steps whereas {\ENERGY} model has difficulties to accept instances exceeding $20$ stations and $20$ minutes times steps within reasonable time.
Most of the time, combinatorial issues make the solver unable to converge.
Improving the heuristic proposed in Chapter \ref{chap:slp} may help to initialise and start the exact searching method with a feasible solution.
Moreover, since the solver is using a Branch-And-Cut method to deal with MIP problems, propose better node evaluations during the searching routine can be a promising feature.



% Clearly, determine the complexity of this problem can provide interesting results for its computation 
%, the complexity of the {\SDP} problem 




% 
% In a future work, we aim to ...

% Putting these features together in addition to a good graphical simulation tool can offer an excellent decision tool for the carsharing decision makers.

% In the last part of the thesis, we presented and evaluated 



%\newpage
%\addcontentsline{toc}{section}{Bibliography of chapter \thechapter}
%\renewcommand{\bibname}{Bibliography of chapter \thechapter}
%\putbib[bib/biblio]
%\end{bibunit}
