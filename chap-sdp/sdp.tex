\chapter{The System Dimensioning Problem} \label{chap:sdp}
\epigraph{test epigraph...}{\textit{in a book \\ by an author}}
\begin{bibunit}[ieeetr]
\minitoc
\vspace{2cm}

\begin{minipage}[c]{0.3\linewidth}
\includegraphics[width=\textwidth]{systemDimensionning}
\end{minipage}
\hfill
\begin{minipage}[c]{0.7\linewidth}
\begin{abstract}
%This section aims at modelling our optimization problem using an integer linear program.
%The inputs and the outputs of the problem are first described.
%Second subsection is dedicated to the building of an oriented valued graph, namely the Timed Extended Graph.
%This graph, previously introduced by Ahuja {\em et al.} \cite{ahujaNetwork1993}, allows to express all the constraints of the problem following the time and the space dimensions.
%The third subsection introduces the decision variables of our optimization problem. More precisely, it is shown that vehicles can be equivalently aggregated into flows to express all the constraints and the criteria of our optimization problem.
%Last subsection presents its formulation using integer linear programming.
The abstract
\end{abstract}
\end{minipage}

\newpage
\section{Introduction : problem description}
The first part of this thesis is dedicated to the optimal design of a carsharing system when stations are fixed.
We assume that a time-dependant carsharing demand can be estimated from origins to destinations.
This demand covers a geographical area where a carsharing operator is interested to established its service.
Without any doubt, the demand estimation model as well as the time period considered in the study are critical inputs in the decision process.
This bias will be discussed afterwards.

In this work, the meaning of optimality will be considered as the will to capture the maximum number of demands.
Somehow, this objective is aligned with the maximization of the operator's revenues.
Indeed, the more demands the systems is able to fulfil, the more profits the operator can generate.

As seen in \cite{jorge_testing_2012}, the optimization process should also integrate vehicle relocation operations.
They are the only known "technique" to re-equilibrate the system and reach a higher satisfied demand.

In the reminder of this work, this system design problem will be refer as the \emph{System Dimensioning Problem} (SDP).
Its concise formulation follows:
\begin{quote}
``Considering potential one-way carsharing station locations and demands over time, what is the optimal \emph{system configuration} capturing the higher number of demands ?''
\end{quote}

We define a system configuration as the quantification and/or the description of the following components:
\begin{enumerate}
\item the number of demands satisfied;
\item the number of vehicles needed to run the system;
\item the vehicle relocation operations.
\end{enumerate}

This chapter is organised as follow.
A first section is devoted to a brief state of the art and related models found in the literature.
Then, Section \ref{sec:mathModel} describes the mathematical materials and defines the problem mathematically.
Section 

% finir l'intro : organisation du chapitre

\section{Related work}

The recent emergence of vehicle-sharing systems have greatly stimulated both industrial and academic research.
In the last chapter, various topics from demand modelling to system management have been discussed.
This section is dedicated to the actual knowledge about one-way carsharing system design and performance.
More precisely, we focus on methodologies extracting design tendencies or structural configurations that may contribute helping carsharing companies to plan their service.

\medskip
Popular approaches facing those problems are based on simulation, micro models or linear programming.
\cite{li_design_2011} presented a discrete-event simulation model to study the performance of one-way carsharing systems.
The model simulates many system operations, including vehicle's pick-up and drop-off, an online reservation system and minimal vehicle relocations.
The number of stations and vehicles is fixed during the simulation.
The global performance of the system is evaluated through various indicators such as the average vehicle utilization, reservation acceptance rate, full parking time and operator profit.
Results demonstrate that system concentricity (density of stations) and the number of parking places are important parameters improving the reservation acceptance rate and vehicle utilisation.


\medskip
More recently, an agent-based simulation model developed by \cite{barrios_performance_2012} aimed to measure and predict the level of service offered to users.
In this study, the carsharing scheme was assumed to be a free-floating and designed to be compared with existing services in Austin, Texas and San Diego, California.
Using agent-based modelling, the author intended to observe emergent phenomena, integrating stochastic demand and users interactions.
The level of accessibility (percentage of an area that is within walking distance of an available carsharing vehicle) was proposed as a metric of the system efficiency.
Simulation results show that the model reasonably estimates the level of service in one-way service context.
The author concluded that the model could help decision makers design the system.


\medskip
Other studies, focussed on optimal policies to relocate vehicles through the system.
As pointed out before, those operations are necessary to ensure that vehicles are located where they are needed.
It allows the system to run at a level of efficiency unreachable otherwise \cite{mitchell_reinventing_2010, nair_fleet_2011, febbraro_one_2012}.

Although not exactly aligned with the purpose of dimensioning a carsharing service, several approaches present interesting models holding suitable system dynamics.
Using mathematical programming, some studies tackle the problem from an operator-based perspective \cite{kek_relocation_2006, kek_decision_2009} whereas others used a station-location approach \cite{correia_optimization_2012, jorge_testing_2012}.
Rather suited for the system dynamics studies, they all rely on a mathematical representation of the whole system based on time expanded graphs \cite{ahuja_network_1993}.
The purpose was to identify measures to maximize resources and enhance service levels.
Results show that the operator profit can be improve when the system runs with suitable vehicle relocation policies.



\section{Mathematical model} \label{sec:mathModel}
\subsection{Formal problem statements}
\paragraph{Inputs:}
% In order to describe the mathematical problem, let us define first the required inputs.
% As discussed later, those data can be extracted from various sources such as simulation tools or real operating carsharing system data for example. 

% time
As seen previously, the SDP is time-dependant.
A carsharing system evolves over time.
Vehicles are moving among stations, taking more or less time to join a destination according to their departure time.
As a consequence, a formal description of the time component is first needed.
In this work, we decided to consider a discrete set of time-steps, numbered from $1$ to $\nbTimeSteps \in \N$ and denoted by $\timeStepSet$.
The time period between consecutive steps, noted $\Delta t$, is set constant and expressed in minutes.
For relevance issues, and especially for experiment purposes, $\timeStepSet$ has to cover a representative period of time, as an average weekday or an average week for instance.
For the record, note that a $24$h period segmented into $10$ minutes time-steps represents a total of $144$ steps (\ie $\nbTimeSteps = 144$).
Some discussions about the assessment of the time component are given in the next chapter.


\medskip
% stations and capacities
Carsharing stations are gathered in a set called $\stationSet$, and also numbered with natural integers from $1$ to $\nbStations > 1$.
Each station comes with a natural integer value $Z(i) > 0$, the maximum capacity (in terms of number of parking places) of station $i \in \stationSet$.
The set of stations' capacities is denoted by $\sCapaSet$.
\medskip

% demands
Then, $\demandSet$ denotes a set of $\nbDemands$ carsharing demands.
A demand $D(i,j,t)$ contains the number of passengers wishing to join station $j \in \stationSet$ from station $i \in \stationSet$, when the departure time at $i$ is $t \in \timeStepSet$.
Note here that there is no condition on the station themselves, thus allowing the potential travel to be a ``round trip'' or a ``one-way'' travel.
In this latter case, $i$ is simply equal to $j$.
\medskip

% travel times
Similarly, $\delta(i,j,t)$ represents the travel time it takes for a car-user picking up a vehicle at station $i \in \stationSet$ to reach station $j \in \stationSet$, when departure time from $i$ is $t \in \timeStepSet$.
Usually, travel times are expressed in minutes, but for the sake of clarity, this value will be converted in number of time-steps as previously defined in $\timeStepSet$.
More generally, any travel time $tt_{min}$ expressed in minutes can be converted using Equation \ref{eqn:travelTimeConversion}.
\begin{equation}\label{eqn:travelTimeConversion}
\delta(i,j,t) = \ceil*{\frac{tt_{min}}{\Delta t}}
\end{equation}

For instance, if $\Delta t = 10$, then a $43$ minutes travel will corresponds to a $5$ time-steps travel.
We suppose that for any triple $(i,j,t) \in \stationSet^2 \times \timeStepSet$, $\delta(i,j,t) < \nbTimeSteps$.
This assumption comes from the fact that the highest distance from different stations is quite low (usually less than $150$ or $200$ kilometres) and that the time-steps are covering at least a day.
\medskip

Finally, the following list summarizes the problem's inputs:
\begin{itemize}
\item $\timeStepSet$: the set of $\nbTimeSteps$ time-steps;
\item $\stationSet$: the set of $\nbStations$ carsharing stations;
\item $\sCapaSet$: the stations' maximum capacities;
\item $\demandSet$: the set of carsharing demands among stations and over time;
\item $\relocTimeStepSet$: the set of time-steps when the system can operates vehicle relocation operations;
\item $\travelTimeSet$: the set of travel times expressed in time-steps.
\end{itemize}

\paragraph{Outputs:}
A carsharing system dimensioning consists in quantifying the system components in order to get a realistic and runnable system.
To be more specific, dimensioning a carsharing system expects the following output values:
\begin{itemize}
\item a set of satisfied demands;
\item a set of vehicle relocation to be operate during the period;
\item the number of parking places in each station;
\item the total number of vehicles.
\end{itemize}
% définir ici ce qu'est une 'optimal system configuration' > nb voitures, nb min de place dans chaque stations...

\paragraph{Statement:}
A formal definition of our main optimization problem, referred as the System Dimensioning Problem with vehicle relocation operations, noted {\SDP} is stated in the following.

\noindent{\SDP}:
\begin{description}
\item[Inputs:] A set of $\nbStations$ stations with their maximum capacity $Z(i) \in \sCapaSet$, $i \in \stationSet$, time periods set ${\timeStepSet = \{1, \cdots, \nbTimeSteps\}}$, travel times $\delta(i,j,t) \in \travelTimeSet$ for each triplet $(i,j,t)\in \stationSet^2 \times \timeStepSet$, a set $\demandSet$ of $\nbDemands$ time-dependant demands, fixed number of vehicles $\nbVehicles$ and relocation operations $\nbVROs$.
\item[Question:]
What is the maximum number of demands $d \leq \nbDemands$ that can be captured by a vehicle routing of at most $\nbVehicles$ vehicles and $\nbVROs$ vehicle relocation operations during the considered period $\timeStepSet$?
\end{description}
We refer later on at SDP($\timeStepSet, \stationSet, \sCapaSet, \demandSet, \relocTimeStepSet, \travelTimeSet, \nbVROs, \nbVehicles$) to indicate a particular SDP instance.
We will show that {\SDP} belongs to ${\cal NP}$ by modelling feasible solutions as a non classical flow problem.

\subsection{Time Expanded Graphs}
In the Station Dimensioning Problem, time is an essential ingredient.
To deal with discrete-time dynamic networks, \cite{ahuja_network_1993} suggested the use of \emph{time-space networks}, also referred as \emph{Time Expanded Graphs} (TEGs). % Ahuja {\em et al.} 
They are static networks constructed by expanding the original (dynamic) networks in the time dimension.
The strength of such models arises from the fact that we benefit from results of static flow models while account properly for the evolution of the underlying system over time.
The basic idea is to consider a separate copy of every node at every discrete time-step. % allows to describe the system over time
Arcs that link these static ``snapshots'' of the network describe temporal linkage in the system.

\medskip
The resulted TEG is a valued directed graph describing the original system over time.
According to conventional notations, a TEG will be denoted by $\teg = (\tegNodeSet, \tegArcSet, \tegCapacity)$ where $\tegNodeSet$, $\tegArcSet$ and $\tegCapacity$ are respectively the set of nodes, the set of arcs and the arcs' capacities.
Next sections aim to describe the graph components, the methodology to build them and their interpretation in the carsharing context.

% However, this time expansion of the original network could become an issue for big instances, especially for the graph density.

\subsubsection{Set of nodes}

\begin{figure}[t]
\begin{minipage}{.65\textwidth}
\begin{center}
\begin{tikzpicture}[>=stealth, thick, scale=2]

\def\timeAxisY{0.2}

% TIME LINE
\draw[timeline] (0.8,\timeAxisY) -- (5.2,\timeAxisY) node[below right]{time};
\foreach \text / \pos in {{$1$/(1,\timeAxisY)}, {$2$/(2,\timeAxisY)}, {$t$/(3.5,\timeAxisY)}, {$\nbTimeSteps$/(5,\timeAxisY)}}
\node[label=below:\text] at \pos {$|$};

\node[label=below:$\cdots$, below=.2cm] at (2.75,\timeAxisY){};
\node[label=below:$\cdots$, below=.2cm] at (4.25,\timeAxisY){};

\foreach \x in {1,2,3.5,5}
\path[timestepline] (\x,\timeAxisY) -- (\x,3.5);

% STATIONS
\foreach \name / \pos / \text in {{station1_1/(1,3)/$1(1)$}, {stationETC/(1,2)/$\vdots$}, {stationS_1/(1,1)/$\nbStations(1)$}, {station1_2/(2,3)/$1(2)$}, {stationETC1/(3.5,3)/$1(t)$}, {station1_T/(5,3)/$1(\nbTimeSteps)$}, {stationS_2/(2,1)/$\nbStations(2)$}, {stationETCS/(3.5,1)/$\nbStations(t)$}, {stationS_T/(5,1)/$\nbStations(\nbTimeSteps)$}}
\node[station] (\name) at \pos {\text};

\end{tikzpicture}
\end{center}
\end{minipage}
\begin{minipage}{.3\textwidth}
In this figure, $\nbTimeSteps$ time-steps are dispatched horizontally whereas the $\nbStations$ stations are distributed vertically.

\bigskip
The graph accounts for a total of $\nbStations \cdot \nbTimeSteps$ nodes and each one of them represents a station's state at a given time-step.
\end{minipage}
\caption{The TEG's node configuration}
\label{fig:tegNodes}
\end{figure}

Nodes in the TEG stand for carsharing stations' states over time.
They are couples $(i,t)$ where $i \in \stationSet$ and $t \in \timeStepSet$ depict the state of each station $i$ at time $t$.
\medskip

As shown in Figure \ref{fig:tegNodes}, the $\nbStations$ stations are represented vertically and duplicated over time following the horizontal time axis.
We denoted by $\tegNodeSet$ the set of nodes in the TEG formally defined as:
\begin{equation}
\tegNodeSet = \stationSet \times \timeStepSet = \{x=(i,t) \mid i \in \stationSet \text{ and } t \in \timeStepSet\}
\end{equation}
The total number of node in $\teg$ is $|\tegNodeSet| = \nbStations \cdot \nbTimeSteps$.
\medskip

Let $\eta$ and $\theta$ the functions which reciprocally return the station number and the time-step associated with every element of $\tegNodeSet$:
\begin{align}
\eta \colon \tegNodeSet  \to \stationSet \text{ ~~~~with~~~~ }  x = (i,t) \mapsto \eta(x) = i \\
\theta \colon \tegNodeSet \to \timeStepSet \text{ ~~~~with~~~~ } x = (i,t) \mapsto \theta(x) = t
\end{align}

\subsubsection{Set of arcs}
The set of arcs is denoted by $\tegArcSet$.
An arc is a pair of nodes picked in $\tegNodeSet$.
It depicted a vehicle \emph{movement} through time.
It is formally defined as the following equation.
\begin{equation}
\tegArcSet \subseteq \{a=(x,y) \mid x \in \tegNodeSet \text{ and } y \in \tegNodeSet\}
\end{equation}

In this study, we consider three temporal operations among stations: park a vehicle in a station, satisfy a carsharing demand and relocate a vehicle.
Those operations are modelled with arcs in $\teg$ and partitioned into tree distinct sets denoted by $\tegArcSet_1$, $\tegArcSet_2$ and $\tegArcSet_3$.

\paragraph{Stay Parked:}
$\tegArcSet_1$ is the set of arcs representing the possibility for the vehicles to stay parked at a station between two consecutive time-steps.
Formally, every arc $a = (x,y) \in \tegArcSet_1$ checks the property $P_1$ defined as:
\begin{numcases}{P_1(a) : }
\eta(x) = \eta(y) & ~ \label{eqn:sameStation}\\
\theta(y) = \theta(x) + 1 \mbox{ mod } \nbTimeSteps & ~ \label{eqn:consecTT}
\end{numcases}

Equation (\ref{eqn:sameStation}) ensures that nodes $x$ and $y$ model the same carsharing station whereas equation (\ref{eqn:consecTT}) expresses the consecutive time-step relation between those station states.

\begin{figure}[t]
\begin{center}
\begin{tikzpicture}[>=stealth, thick, scale=2]

\def\timeAxisY{0.5}

% TIME LINE
\draw[timeline] (0.8,\timeAxisY) -- (5.2,\timeAxisY) node[below right]{time};
\foreach \text / \pos in {{$1$/(1,\timeAxisY)}, {$2$/(2,\timeAxisY)}, {$t$/(3.5,\timeAxisY)}, {$\nbTimeSteps$/(5,\timeAxisY)}}
\node[label=below:\text] at \pos {$|$};

\node[label=below:$\cdots$, below=.2cm] at (2.75,\timeAxisY){};
\node[label=below:$\cdots$, below=.2cm] at (4.25,\timeAxisY){};

\foreach \x in {1,2,3.5,5}
\path[timestepline] (\x,\timeAxisY) -- (\x,1.5);

% STATIONS
\foreach \name / \pos / \text in {{station1_1/(1,1)/$1(1)$}, {station1_2/(2,1)/$1(2)$}, {stationETC1/(3.5,1)/$1(t)$}, {station1_T/(5,1)/$1(\nbTimeSteps)$}}
\node[station] (\name) at \pos {\text};

% ARCS STOCK
\path[arcStock] (station1_1) -> (station1_2);
\path[arcStock] (station1_2) -> (stationETC1) node[midway, fill=white] {$\cdots$};
\path[arcStock] (stationETC1) -> (station1_T) node[midway, fill=white] {$\cdots$};
\path[arcStock] (station1_T) edge [in=30, out=150] (station1_1);

\end{tikzpicture}
\end{center}
\caption{Stock arcs in the TEG}
\label{fig:tegStockArcs}
\end{figure}

\medskip
Figure \ref{fig:tegStockArcs} illustrates a single station over time where parking a vehicle is model with stock arcs.
Note here that an arc exists between node $1(\nbTimeSteps)$ and $1(1)$.
In our model, time-steps number $\nbTimeSteps$ and $1$ are considered as consecutive.
The total number of arc in $\tegArcSet_1$ is exactly the number of nodes in the graph, \ie ${|\tegArcSet_1| = |\tegNodeSet| = \nbStations \times \nbTimeSteps}$, since every node in $\teg$ have a unique adjacent stock arc in $\tegArcSet_1$.

\paragraph{Fulfil a demand:}
Arcs in $\tegArcSet_2$ are associated with a vehicle demand.
Each demand $D(i,j,t) > 0$ given as input is modelled with an arc linking the station $i$ at time-step $t$, \ie node $x = (i,t) \in \teg$, to station $j$.
The arrival time-step at station $j$ is calculated using the known travel time $\delta(i,j,t)$ given as input and its value is $t + \delta(i,j,t) \mbox{ mod } \nbTimeSteps$.
Formally, every arc $a = (x,y)$ belongs to $\tegArcSet_2$ if it fits the following property $P_2$:

\begin{numcases}{P_2(a) : }
D(\eta(x), \eta(y), \theta(x)) > 0 & ~\\
\theta(x) + \delta(\eta(x), \eta(y), \theta(x)) = \theta(y) \mbox{ mod } \nbTimeSteps & ~
\end{numcases}

Figure \ref{fig:tegDemandArcs} illustrates the construction of the demand arcs.
In this basic example, two demands, ${D_1 = D(1,2,1)}$ and ${D_2 = D(2,1,\nbTimeSteps)}$ are represented respectively by arc ${a_1 = (x_1, y_1)}$ and arc ${a_2 = (x_2, y_2)}$.
Using travel times ${\delta_1 = \delta(1,2,1) = 1}$ and ${\delta_2 = \delta(2,1,\nbTimeSteps) = 2}$ as inputs, the couple of arcs is build as follows:

\begin{minipage}{.45\textwidth}
\begin{numcases}{}
x_1 = (1, 1) & ~ \nonumber\\
y_1 = (2, 1 + \delta_1 \mbox{ mod } \nbTimeSteps = 2) = (2, 2) & ~ \nonumber
\end{numcases}
\end{minipage}
\hfill
\begin{minipage}{.45\textwidth}
\begin{numcases}{}
x_2 = (2, \nbTimeSteps) & ~ \nonumber\\
y_2 = (1, \nbTimeSteps + \delta_2 \mbox{ mod } \nbTimeSteps = 2) = (1, 2) & ~ \nonumber
\end{numcases}
\end{minipage}

\begin{figure}[t]
\begin{center}
\begin{tikzpicture}[>=stealth, thick, scale=2]

\def\timeAxisY{0.5}

% TIME LINE
\draw[timeline] (0.8,\timeAxisY) -- (5.2,\timeAxisY) node[below right]{time};
\foreach \text / \pos in {{$1$/(1,\timeAxisY)}, {$2$/(2,\timeAxisY)}, {$t$/(3.5,\timeAxisY)}, {$\nbTimeSteps$/(5,\timeAxisY)}}
\node[label=below:\text] at \pos {$|$};

\node[label=below:$\cdots$, below=.2cm] at (2.75,\timeAxisY){};
\node[label=below:$\cdots$, below=.2cm] at (4.25,\timeAxisY){};

\foreach \x in {1,2,3.5,5}
\path[timestepline] (\x,\timeAxisY) -- (\x,2.5);

% STATIONS
\foreach \name / \pos / \text in {{station1_1/(1,2)/$1(1)$}, {station1_2/(2,2)/$1(2)$}, {stationETC1/(3.5,2)/$1(t)$}, {station1_T/(5,2)/$1(\nbTimeSteps)$}, {station2_1/(1,1)/$2(1)$}, {station2_2/(2,1)/$2(2)$}, {stationETC2/(3.5,1)/$2(t)$}, {station2_T/(5,1)/$2(\nbTimeSteps)$}}
\node[station] (\name) at \pos {\text};

% ARCS DEMAND
\path[arcDemand] (station1_1) edge[in=120, out=-45] node[above]{$a_1$} (station2_2);
\path[arcDemand] (station2_T) edge[in=-45, out=135] node[near start, above]{$a_2$} (station1_2);

\end{tikzpicture}
\end{center}
\caption{Graph representation of two arc demands}
\label{fig:tegDemandArcs}
\end{figure}

\paragraph{Relocate a vehicle between stations:}
$\tegArcSet_3$ represents all the possible vehicle relocation operations over time.
Let define $\relocTimeStepSet \subseteq \timeStepSet$\label{def:relocTimeStepSet}, the subset of time-steps when the system could be balanced through vehicle relocation operations.
Then it comes that any relocation arc can be model as arcs starting from every station at every time-steps in $\relocTimeStepSet$.

\medskip
Similarly to demand arcs, the arrival nodes are calculated taking into account travel times between stations given as input.
Formally, any triple $(i,j,t) \in \stationSet^2 \times \relocTimeStepSet$ with $i \neq j$ is modelled as a vehicle relocation arc connecting node $x = (i,t)$ to node $y = (j, t + \delta(i,j,t) \mbox{ mod } \nbTimeSteps)$.
Property $P_3$ ensures that any arc $a = (x, y)$ belongs to $\tegArcSet_3$.

\begin{numcases}{P_3(a) : }
\eta(x) 										\not= \eta(y)					& ~\\
\theta(x) + \delta(\eta(x), \eta(y), \theta(x)) = \theta(y)  \mbox{ mod } \nbTimeSteps		& ~
\end{numcases}

\begin{figure}[t]
\begin{center}
\begin{tikzpicture}[>=stealth, thick, scale=2]

\def\timeAxisY{0.2}

% TIME LINE
\draw[timeline] (0.8,\timeAxisY) -- (5.2,\timeAxisY) node[below right]{time};
\foreach \text / \pos in {{$1$/(1,\timeAxisY)}, {$2$/(2,\timeAxisY)}, {$t$/(3.5,\timeAxisY)}, {$\nbTimeSteps$/(5,\timeAxisY)}}
\node[label=below:\text] at \pos {$|$};

\node[label=below:$\cdots$, below=.2cm] at (2.75,\timeAxisY){};
\node[label=below:$\cdots$, below=.2cm] at (4.25,\timeAxisY){};

\foreach \x in {1,2,3.5,5}
\path[timestepline] (\x,\timeAxisY) -- (\x,3.5);

% STATIONS
\foreach \name / \pos / \text in {{station1_1/(1,3)/$1(1)$}, {stationETC/(1,2)/$\vdots$}, {stationS_1/(1,1)/$\nbStations(1)$}, {station1_2/(2,3)/$1(2)$}, {stationETC1/(3.5,3)/$1(t)$}, {station1_T/(5,3)/$1(\nbTimeSteps)$}, {stationS_2/(2,1)/$\nbStations(2)$}, {stationETCS/(3.5,1)/$\nbStations(t)$}, {stationS_T/(5,1)/$\nbStations(\nbTimeSteps)$}}
\node[station] (\name) at \pos {\text};

% ARCS RELOC
\foreach \v in {{(1.1,1.6)}, {(1.4,1.4)}, {(1.2,.9)}}
{
\draw (stationS_1) edge[arcReloc] ++\v;
\draw (stationS_2) edge[arcReloc] ++\v;
\draw (stationETCS) edge[arcReloc] ++\v;
}
\end{tikzpicture}
\end{center}
\caption{Demand arcs in the TEG}
\label{fig:tegRelocArcs}
\end{figure}

Figure \ref{fig:tegRelocArcs} depicts the set of relocation arcs in a complete TEG where $\relocTimeStepSet = \timeStepSet$.
At every time-step, every station is connected to the other.
In this model, a vehicle relocation operation can be performed at any time between every pair of stations.
The total number of vehicle relocation arcs can be calculated using equation (\ref{eqn:nbVehicleRelocationOperation}).
\begin{equation}\label{eqn:nbVehicleRelocationOperation}
|\tegArcSet_3| = |\relocTimeStepSet| \cdot \nbStations \cdot (\nbStations - 1)
\end{equation}



\bigskip
Finally, sets $\tegArcSet_1$, $\tegArcSet_2$, $\tegArcSet_3$ are then formally defined as:
\begin{equation}
\tegArcSet_k = \{x \in \tegNodeSet \mid P_k(x)\},\quad\forall k \in \{1, 2, 3\}.
\end{equation}
Together, they form a mathematical partition of the set of arcs.
Assuming that $\demandSet \not= \emptyset$, the three subsets hold the following relations:
\begin{enumerate}
\item $\tegArcSet = \bigcup\limits_{k=1}^3 \tegArcSet_k$;
\item $\tegArcSet_i \cap \tegArcSet_j = \emptyset, \quad\forall i,j \in \{1,2,3\}, \quad i\not= j$.
\end{enumerate}
The total number of arcs is given by:
\begin{equation} \label{eqn:tegNbArcs}
|\tegArcSet| = \sum_{k=1}^3 |\tegArcSet_k| = \nbStations \cdot \nbTimeSteps + \nbDemands + |\relocTimeStepSet| \cdot \nbStations \cdot (\nbStations - 1)
\end{equation}
where $\nbDemands$ is the number of requested demands.
When $\relocTimeStepSet = \timeStepSet$, and since $\nbDemands \ll \nbStations^2$, then it comes that ${|\tegArcSet| = \Theta(\nbStations^2 \cdot \nbTimeSteps)}$.
In this case, we observe that $|\tegArcSet_3|\gg |\tegArcSet_1 \cup \tegArcSet_2|$ and that the number of arcs is then proportional to $\tegArcSet_3$.

\medskip
Figure \ref{fig:teg} illustrates a complete TEG configuration with $\nbStations$ stations, $\nbTimeSteps$ time-steps and $\relocTimeStepSet = \timeStepSet$.
\begin{figure}[t]
\centering
\begin{tikzpicture}[>=stealth, thick] % shorten >=1pt

% AXE TEMPOREL
\node[dumbnode] (dumb1) at (0,-3.2) {};

\node[graduation] (markt0) [right of = dumb1, node distance = .5cm] {$|$};
\node[graduation] (markt1) [right of = markt0, node distance = 2cm] {$|$};
\node[graduation] (markt2) [right of = markt1, node distance = 2cm] {$|$};
\node[graduation] (markt3) [right of = markt2, node distance = 2cm] {$|$};
\node[timestep] (t0) [below of = markt0] {$1$};
\node[timestep] (t1) [below of = markt1] {$2$};
\node[timestep] (cdots_2t) [right of = t1, node distance = 1cm] {$\cdots$};
\node[timestep] (t2) [below of = markt2] {$t$};
\node[timestep] (cdots_tT) [right of = t2, node distance = 1cm] {$\cdots$};
\node[timestep] (t3) [below of = markt3] {$\nbTimeSteps$};

\node[dumbnode] (dumb2) [right of = markt3, node distance = 1cm] {};

\path[timeline] (dumb1) -- (dumb2);
\node[dumbnode] (time) [below right of = dumb2, node distance = .4cm] {time};

% LIGNES POINTILLEES VERTICALES POUR LES INSTANTS TEMPORELS
\node[dumbnode] (dumb3) [above of = markt0, node distance = 7.2cm] {};
\node[dumbnode] (dumb4) [right of = dumb3, node distance = 2cm] {};
\node[dumbnode] (dumb5) [right of = dumb4, node distance = 2cm] {};
\node[dumbnode] (dumb6) [right of = dumb5, node distance = 2cm] {};
  
\path[timestepline] (markt0) -- (dumb3);
\path[timestepline] (markt1) -- (dumb4);
\path[timestepline] (markt2) -- (dumb5);
\path[timestepline] (markt3) -- (dumb6);
  
% STATIONS
\node[station] (station1_1) [above of = markt0, node distance = 5.5cm] {1(1)};
\node[station] (stationETC) [below of = station1_1, node distance = 2cm] {$\vdots$};
\node[station] (stationN_1) [below of = stationETC, node distance = 2cm] {$\nbStations(1)$};

\node[station] (station1_2) [right of = station1_1, node distance = 2cm] {1(2)};
\node[station] (stationETC1) [right of = station1_2, node distance = 2cm] {$\mathbf{\cdots}$};
\node[station] (station1_T) [right of = stationETC1, node distance = 2cm] {$1(\nbTimeSteps$)};
\node[station] (stationN_2) [right of = stationN_1, node distance = 2cm] {$\nbStations(2)$};
\node[station] (stationETCN) [right of = stationN_2, node distance = 2cm] {$\mathbf{\cdots}$};
\node[station] (stationN_T) [right of = stationETCN, node distance = 2cm] {$\nbStations(\nbTimeSteps)$};

% ARCS STOCK
\path[arcStock] (station1_1) edge node[capacity]{(8)} (station1_2);
\path[arcStock] (station1_2) edge node[capacity]{(8)} (stationETC1);
\path[arcStock] (stationETC1) edge node[capacity]{(8)} (station1_T);
\path[arcStock] (station1_T) edge [bend right, in=220, out=320] node[capacity]{(8)} (station1_1);

\path[arcStock] (stationN_1) edge node[capacity]{(4)} (stationN_2);
\path[arcStock] (stationN_2) edge node[capacity]{(4)} (stationETCN);
\path[arcStock] (stationETCN) edge node[capacity]{(4)} (stationN_T);
\path[arcStock] (stationN_T) edge [bend left] node[swap, capacity]{(4)} (stationN_1);

% ARCS DEMAND
\path[arcDemand] (station1_1) edge [bend left, out=5] node[capacity, left, near start]{(1)} (stationN_2);
\path[arcDemand] (stationN_T) edge [bend right, out=15] node[swap, capacity, near start]{(2)} (station1_2);

% ARCS RELOC
\node[dumbnode] (dumb7) [below right of = station1_2, node distance = 1cm] {};
\node[dumbnode] (dumb8) [below right of = dumb7, node distance = .7cm] {};
\node[dumbnode] (dumb9) [below of = dumb8, node distance = .5cm] {};

\path[arcReloc] (stationN_1) edge (station1_2.south);
\path[arcReloc] (stationN_1) edge (dumb7.center);
\path[arcReloc] (stationN_1) edge (dumb8.center);
\path[arcReloc] (stationN_1) edge (dumb9.center);

\node[dumbnode] (dumb10) [below right of = stationETC1, node distance = 1cm] {};
\node[dumbnode] (dumb11) [below right of = dumb10, node distance = .7cm] {};
\node[dumbnode] (dumb12) [below of = dumb11, node distance = .5cm] {};

\path[arcReloc] (stationN_2) edge (stationETC1.south);
\path[arcReloc] (stationN_2) edge (dumb10.center);
\path[arcReloc] (stationN_2) edge (dumb11.center);
\path[arcReloc] (stationN_2) edge (dumb12.center);

% LEGENDE\\
\node[station] (stationLeg) [right of = station1_T, node distance = 2cm, anchor = west] {i(t)};
\node[dumbnode] (stationLegText) [right of = stationLeg, node distance = .76cm, anchor = west] {Station $i$ at time $t$};

\node[dumbnode] (legArcStock1) [below left of = stationLeg, node distance = 1.2cm, anchor = west] {};
\node[dumbnode] (legArcStock2) [right of = legArcStock1, node distance = 1.5cm, anchor = west] {Stock arcs: $({\cal A}_1)$};
\path[arcStock] (legArcStock1) -- (legArcStock2);

\node[dumbnode] (legArcDem1) [below of = legArcStock1, node distance = .5cm] {};
\node[dumbnode] (legArcDem2) [right of = legArcDem1, node distance = 1.5cm, anchor = west] {Demand arcs: $({\cal A}_2)$};
\path[arcDemand] (legArcDem1) -- (legArcDem2);

\node[dumbnode] (legArcReloc1) [below of = legArcDem1, node distance = .5cm] {};
\node[dumbnode] (legArcReloc2) [right of = legArcReloc1, node distance = 1.5cm, anchor = west] {Relocalisation arcs: $({\cal A}_3)$};
\path[arcReloc] (legArcReloc1) -- (legArcReloc2);

\node[dumbnode] (legcap) [below of = legArcReloc1, node distance = .5cm, anchor = west] {$(\cdot)$};
\node[dumbnode] (legca2) [right of = legcap, node distance = 1.2cm, anchor = west] {Arc capacity};

% SUITE / DENSITES
\node (densiteTEG) [below of = legArcReloc1, node distance = 2.5cm, anchor = west, align=left]{
Graph density :\\
Nodes : $|\tegNodeSet| = \nbStations \times \nbTimeSteps$\\
Arcs : $|\tegArcSet| = \nbStations^2 \cdot \nbTimeSteps + \nbDemands = \Theta(\nbStations^2 \cdot \nbTimeSteps)$};
\end{tikzpicture}
\caption{Complete TEG with $\nbStations$ stations and $\nbTimeSteps$ time-steps. Maximum capacities for stations $1$ and $\nbStations$ are respectively $8$ and $4$. The two demands are the same as those in figure (\ref{fig:tegDemandArcs})(1.3).}
\label{fig:teg}
\end{figure}

\subsubsection{Arcs Capacities}
Associated with each arc $a = (x,y) \in \tegArcSet$, is given a capacity function $u: \tegArcSet \rightarrow \N$ corresponding to a maximum number of vehicles allowed on $a$.
It is defined as follows for any arc $a \in \tegArcSet$:
\begin{numcases}{\tegCapacity(a) =}
Z(\eta(x))                    & if $a \in \tegArcSet_1$, \nonumber \\
D(\eta(x), \eta(y),\theta(x)) & if $a \in \tegArcSet_2$, \nonumber \\
+ \infty                      & if $a \in \tegArcSet_3$. \nonumber
\end{numcases}

For any arc $a = (x,y) \in \tegArcSet_1$, the maximum number of cars is the capacity of the station $\eta(x) = \eta(y)$.
It corresponds to the demand for any arc $a \in \tegArcSet_2$, and it is not bounded for relocation arcs $a \in \tegArcSet_3$.

% TEG example
%\begin{figure}[!t]
%\centering
%\includegraphics[scale=0.21]{tikx_arcsStock}
%\caption{Time-Extended-Graph example}
%\label{TEGExample}
%\end{figure}

%The figure \ref{TEGExample} aims to help represent each topology of each sets of arcs. Capacities over arcs of $\tegArcSet_1$ and $\tegArcSet_2$ are represented in brackets, whereas %those infinite of $\tegArcSet_3$ do not appear in the figure. Let's notice that this example don't represent all the arcs of the graph since it is very dense.\\

\subsubsection{Additional notations}
We denote by $\Gamma^-(x)$ and $\Gamma^+(x)$ respectively the set of immediate predecessors and successors of a node $x \in \tegNodeSet$.
\begin{align}
\Gamma^-(x) = \{y \in \tegNodeSet \mid (y,x) \in \tegArcSet\} \\
\Gamma^+(x) = \{y \in \tegNodeSet \mid (x,y) \in \tegArcSet\}
\end{align}

For any couple of time instants $\forall (t, t') \in \timeStepSet^2$, 
the number of time-steps between those two instants is defined through the following function:

\begin{minipage}{.25\textwidth}
\begin{align}
\vartheta : {\timeStepSet}^2 & {\quad \rightarrow \quad} \N \nonumber \\
(t, t') & {\quad \mapsto \quad} \vartheta (t, t') \nonumber
\end{align}
\end{minipage}
\begin{minipage}{.75\textwidth}
\begin{numcases}{\text{with~~~~~~} \vartheta(t, t') =}
t' - t & if $t \leq t'$, \nonumber \\
T + t' - t & otherwise. \nonumber
\end{numcases}
\end{minipage}

\bigbreak
For each arc $a = (x,y) \in \tegArcSet$, let us define the boolean value $\epsilon_a$ as:
\begin{numcases} {\epsilon_a =}
0 & if $\theta(x) \leq \theta(y)$, \nonumber \\
1 & otherwise. \nonumber 
\end{numcases}

The time required for a movement from $x$ to $y$ is then given by the function
\begin{align}
\ell : \tegArcSet & {\quad \rightarrow \quad} \N \\
a = (x, y) & {\quad \mapsto \quad} \ell(a) = \theta(y) - \theta(x) + \epsilon_a \cdot \nbTimeSteps
\end{align}

By extension, if $\mu = (a_1, \cdots, a_p) \in \tegArcSet^p$ is a path in the TEG from $x$ to $y$, the value $\ell(\mu)=\sum_{i=1}^p \ell(a_i)$ is the total time required for a vehicle going from $x$ to $y$ following $\mu$.

For any time value $t \in \timeStepSet$, let us define the set ${\Coupe}_t(\mu)$ as the arcs $a=(x,y)$ from $\mu$ starting at time $t$ or earlier but ending after $t$.
Formally,
\begin{equation}
{\Coupe}_t(\mu) = \{a=(x,y) \in \mu \mid \vartheta(\theta(x),t) < \ell(a)\}.
\end{equation}

\subsection{Flows and decision variables}

Consider vehicles directly in the model does not seem to be a good option.
Although it would be possible to create boolean variables for each node of the TEG to track and follow the movement of vehicles over the day, their number may be too large.
Thus, we first introduce flow variables over arcs in the Time Extended graph.
It is shown then that feasible vehicle tours can be extracted easily from any feasible flow.

\medskip
The aim of our study is to compute the planning of each vehicles during the period. 
At any time, each vehicle is either parked in a station or in transit between two stations.
Its position over the period can be modelled as a vehicle tour \ie a circuit denoted by $c=(a_1,\cdots, a_p)$ in the TEG.
The size of any feasible solution may be highly reduced if we only consider the number of vehicles passing through each arc.  
Each vehicle in the system is thus associated to a vehicle tour (even if the vehicle is not used, staying in a station).

\medskip
The duration of any path or circuit in a TEG may be easily evaluated.
Indeed, for any couple of time instants $(t, t') \in {\cal H}^2$, let the function $\vartheta:{\cal H}^2 \mapsto \mathbb{N}^\star$ that computes the number of time-steps between those two instants, formally defined as follow:
\begin{numcases}{\vartheta(t, t') =}
t' - t & if $t \leq t'$, \nonumber \\
\nbTimeSteps + t' - t & otherwise. \nonumber
\end{numcases}

For each arc $a=(x,y)\in \tegArcSet$, let us define and set the boolean value $\epsilon_a$ to true if  $\theta(x)>\theta(y)$.
The effective time required for a move from  $x$ to $y$ is then equal to $\ell(a) = \theta(y) - \theta(x) + \epsilon_a \cdot \nbTimeSteps$.
By extension, if $\mu= (a_1, \cdots, a_p) \in \tegArcSet^p$ is a path of the TEG from $x$ to $y$, the value $\ell(\mu)=\sum_{i=1}^p \ell(a_i)$ is the total time required for a vehicle going from $x$ to $y$ following $\mu$.
Next lemma evaluates the total time of any circuit $c$.
\begin{lemma} \label{timecircuit}
The total time of any circuit $c = (a_1, \cdots, a_p) \in \tegArcSet^p$ is $\ell(c) = \nbTimeSteps \times \sum\limits_{i=1}^p \epsilon_{a_i}$.
\end{lemma}
\begin{proof}
Let $x_i$, $i \in \{1, \cdots, p+1\}$ be the sequence of elements from $\tegNodeSet$ such that, $x_{p+1} = x_1$ and
$\forall i \in \{1, \cdots, p\}$, $a_i = (x_i, x_{i+1})$.
The total time of $c$ is then
\[\ell(c)=\sum_{i=1}^p \ell(a_i) = \sum_{i=1}^p ( \theta(x_{i+1})- \theta(x_{i}) +\epsilon_{a_i} \cdot\nbTimeSteps) =\nbTimeSteps\times \sum_{i=1}^p \epsilon_{a_i},\]
the result.
\end{proof}

%\subsection{Decision variables (MOSIM)}
%The aim of our modelling is to not consider if possible the vehicles directly to solve our optimization problem. The main reason is that this number may be quite important, and a vehicle tour may be associated to each of them.



%The aim of our study is to compute the planning of each vehicles during the period. 
%At any time, each vehicle is either parked in a station or in transit between two stations. Its position over the period can be modelled as a vehicle tour \ie a circuit $c=(a_1,\cdots, a_p)$ in the TEG.
%
%The size of any feasible solution may be highly reduced if we only consider the number of vehicles passing through each arc.  
%%Each vehicle in the system is thus associated to a vehicle tour (even if the vehicle is not used, staying in a station).
%%\subsubsection{Flow variables}
%For each arc $a =(x,y) \in \tegArcSet$, we call $\varphi(a)$ the flow of vehicles transiting through the arc $a$. It can be interpreted as:
%\begin{itemize}
%\item the number of vehicle staying in station $\eta(x)$ between two consecutive time-steps $\theta(x)$ and $\theta(y)$, if $a \in \tegArcSet_1$;
%\item the number of vehicle picked by users from station $\eta(x)$ at time $\theta(x)$ to station $\eta(y)$, if $a \in \tegArcSet_2$;
%\item the number of vehicle relocated between stations $\eta(x)$ and $\eta(y)$ at time $\theta(x)$, if $a \in \tegArcSet_3$.
%\end{itemize}
%
%The total number of vehicles transiting to any node $x\in \cX$ is clearly constant, thus
%\[\sum_{\substack{y\in \Gamma^-(x)}} \varphi((y,x))= \sum_{\substack{y\in \Gamma^+(x)}} \varphi((x,y)).\]
%
%%\subsubsection{Building vehicles tours}
%
%It is immediate that a feasible flow may be obtained from any feasible set of vehicle tours.
%We prove in the following that vehicle tours may be easily computed from a feasible flow.

%\subsection{Decision variables (EWGT)}
The aim of our study is therefore to compute the planning of each vehicles during the period $\timeStepSet$. 
At any time, each of them is either parked in a station or in transit between two stations.
Its position over the period can be modelled as a vehicle tour \ie a circuit $c=(a_1,\cdots, a_p)$ in the TEG.
We show in the following that a feasible solution can be described by only considering the number of vehicles modelled as a flow passing through each arc.

\medskip
For each arc $a =(x,y) \in \tegArcSet$, we call $\varphi(a)$ the flow of vehicles transiting through the arc $a$. 
It can be interpreted as the number of vehicle staying in station $\eta(x)$ between two consecutive time-steps if $a \in \tegArcSet_1$, or the number of vehicle moving from station $\eta(x)$ at time $\theta(x)$ to station $\eta(y)$ otherwise.
Since the total number of vehicles transiting by any node $x \in \tegNodeSet$ is constant, the flow verifies the following flow conservation equation:
\begin{equation}\label{eqn:flowconservation}
\sum_{{y\in \Gamma^-(x)}} \varphi((y,x))= \sum_{{y\in \Gamma^+(x)}} \varphi((x,y)),
\end{equation}
A flow $\varphi:\tegArcSet \mapsto \N$ is then said to be feasible if $\forall a \in \tegArcSet$, its value do not exceeds its capacity, \ie $\varphi(a) \leq u(a)$ and $\forall x \in \tegNodeSet$, the flow conservation equation (\ref{eqn:flowconservation}) is true.
A feasible flow may be easily obtained from any feasible set of vehicle tours.

\medskip
We prove in the following that the reverse is also true, with the consequence that any feasible solution of our problem can be described using a flow.
Next lemma computes the exact number of vehicles associated to a constant unitary flow over a circuit $c$.

\begin{lemma} \label{NbVoitCircuit}
Let $c$ be a circuit and $\varphi_c$ a feasible flow such that:
\begin{numcases} {\varphi_c(a) =}
1 & if $a$ belongs to $c$, \nonumber \\
0 & otherwise. \nonumber 
\end{numcases}
The minimum number of vehicles to insure $\varphi_c$ is $\frac{\ell(c)}{\nbTimeSteps}$.
\end{lemma}
\begin{proof}
For any time value $t \in \timeStepSet$, let us define the set ${\cal C}_t(c)$ as
the arcs $a=(x,y)$ from $c$ starting at time $t$ or earlier but ending after $t$.
Since $\vartheta(\theta(x), t)$ equals the number of time steps from $\theta(x)$ to $t$, we get ${\cal C}_t(c) = \{a=(x,y) \in c \mid \vartheta(\theta(x), t) < \ell(a)\}$.

\medskip
Now, since $c$ is a circuit, the value $\vert {\cal C}_t(c) \vert$ is a constant $\forall t \in \timeStepSet$ and corresponds to the total number of vehicles needed to insure a unitary flow over $c$.
Let us prove that $\vert {\cal C}_\nbTimeSteps(c) \vert = \sum\limits_{a \in c} \epsilon_a$. 
For that purpose, setting $B(c) = \{a=(x,y) \in c \mid \epsilon_a = 1\}$, we show that $B(c)= {\cal C}_\nbTimeSteps(c)$.
\begin{itemize}
\item $B(c) \subseteq {\cal C}_\nbTimeSteps(c)$:
if $a=(x,y) \in B(c)$, then as $\theta(x) \leq \nbTimeSteps$, $\vartheta(\theta(x), \nbTimeSteps) = \nbTimeSteps - \theta(x)$.
Now, since $\epsilon_a=1$ and $\theta(y)\geq 1$, $\ell(a) = \theta(y) - \theta(x) + \nbTimeSteps \geq 1 - \theta(x) + \nbTimeSteps > \vartheta(\theta(x),\nbTimeSteps)$ and $a\in {\cal C}_\nbTimeSteps(c)$.

\item ${\cal C}_\nbTimeSteps(c) \subseteq B(c)$:
let consider now an arc $a=(x,y) \in {\cal C}_\nbTimeSteps(c)$.
Since $\vartheta(\theta(x), \nbTimeSteps) = \nbTimeSteps - \theta(x) < \ell(a)$ we get that 
$\theta(y) - \theta(x) + \epsilon_a \cdot \nbTimeSteps > \nbTimeSteps - \theta(x)$ and thus 
$\theta(y) + \epsilon_a \cdot \nbTimeSteps > \nbTimeSteps$.  
As $\theta(y) \leq \nbTimeSteps$, we necessarily have $\epsilon_a = 1$ and thus $a \in B(c)$.
\end{itemize}
Now, by Lemma \ref{timecircuit}, $\vert {\cal C}_\nbTimeSteps(c) \vert = \sum\limits_{a\in c} \epsilon_a = \frac{\ell(c)}{\nbTimeSteps}$, the lemma.
\end{proof}

\begin{theorem} \label{th:decomp} % l'ensemble des circuits ${\cal S}$ porte le même nom que l'ensemble des stations... à modifier peut être pour éviter la confusion
Any feasible solution $\varphi$ can be decomposed into a set of circuits ${\cal S}$ such that, for any arc $a \in \tegArcSet$, $\varphi(a) = \sum\limits_{c \in {\cal S}} \varphi_c(a)$.
\end{theorem}
\begin{proof}
The proof is by recurrence on $n(\varphi) = \sum\limits_{a \in \tegArcSet} \varphi(a)$.
The theorem is trivially true if $n(\varphi) = 0$.

Let suppose now that $n(\varphi) > 0$, thus there exists at least one arc $a = (x, y) \in \tegArcSet$ with $\varphi(a) > 0$.
Set $\mu_0 = (x, y)$ and let consider the sequence of paths $\mu_i$ built as follows:
\begin{enumerate}
\item
Stop the sequence as soon as $\mu_i$ contains a circuit $c$,
\item
Otherwise, let $\tilde{a}=(\tilde{x},\tilde{y})$ the last arc of $\mu_i$. 
Since $\varphi(\tilde{a}) > 0$, the flow conservation equation (\ref{eqn:flowconservation}) insures that there exists an arc $a$ starting at $\tilde{y}$ with $\varphi(a)>0$.
We then set $\mu_{i+1}=\mu_i\cdot a$.
\end{enumerate}
As $\teg$ has a finite number of nodes, the algorithm stops and a non empty circuit $c$ is  returned.
The flow $\hat{\varphi}$ defined as
\begin{numcases}
{\hat{\varphi}(a) =}
\varphi(a)-1 &  if $a \in c$, \nonumber \\
\varphi(a) &  otherwise. \nonumber 
\end{numcases}
is feasible with $n(\hat{\varphi}) < n(\varphi)$, thus the theorem.
\end{proof}

Note that the  number of flow variables is a polynomial function on the size of the problem.
This is not true anymore for vehicle tours, which number can be exponential.
The consequence is that the determination of a flow is in ${\cal NP}$, which is not the case for the determination of vehicle tours.

\begin{theorem} \label{NbVoiture}
The minimum number of vehicles required for a feasible flow $\varphi$ equals $\sum\limits_{a\in \tegArcSet} \varphi(a) \cdot \epsilon_a$.
\end{theorem}
\begin{proof}
Let ${\cal S}$ be a set of circuits obtained from the decomposition of $\varphi$ following Theorem \ref{th:decomp} and let $V$ be the minimum number of cars associated with $\varphi$.
By Lemmas \ref{timecircuit} and \ref{NbVoitCircuit}, the total number of car of any circuit $c \in {\cal S}$ is
\[\sum_{a\in c} \epsilon_a = \sum_{a\in \tegArcSet} \epsilon_a \cdot \varphi_c(a)\]
and thus
\[V \leq \sum_{c\in {\cal S}} \sum_{a\in \tegArcSet} \epsilon_a \cdot \varphi_c(a).\]
Now, from  Theorem \ref{th:decomp}, $\varphi (a)= \sum_{c\in {\cal S}}\varphi_c(a)$.
Thus, 
\[\sum_{c\in {\cal S}} \sum_{a\in \tegArcSet} \epsilon_a \cdot \varphi_c(a)=\sum_{a\in \tegArcSet} \epsilon_a \cdot \sum_{c\in {\cal S}}  \varphi_c(a)=\sum_{a\in \tegArcSet} \varphi(a) \cdot \epsilon_a.\]
Lastly, the total number of vehicles required at time $\nbTimeSteps$ to reach $\varphi$ is exactly $\sum_{a\in \tegArcSet} \varphi(a) \cdot \epsilon_a$, the theorem.
\end{proof}

\subsection{Solution and objectives}
A feasible solution of our problem consists on a set of vehicle tours, each of them modelling the situation of a car at each time step.

\medskip
A feasible solution of our optimization problem is given by a set of vehicles, each of them associated with its position in the system at any time-step during the period.
The first objective is to maximize the total number of satisfied demands, \ie for which a vehicle is allocated.
Besides, two other objectives must be taken into account: each vehicle in the system is associated to a fixed cost, so that the total number of vehicles must be minimized. 
In the same way, vehicle relocations are fundamental for increasing the number of satisfied demands with a fixed  number of vehicles.
However, they cost an extra charge for the operator, and thus their number should also be limited.
In the following, the total number of vehicles and relocations are referred respectively by $\nbVehicles$ and $\nbVROs$.
The three criteria (the demand, the number of vehicles and the number of relocation operations) can be polynomially computed from any feasible solution.

%\subsection{How to recover the number of vehicles?}
%Next lemmas characterize the total time of any circuit $c$ and the exact number of vehicles required for a unitary flow on $c$.

\subsection{Mathematical program}



\medskip
The modelling of our optimization problem follows.
$\nbVROs$ and $\nbVehicles$ are fixed integer bounds for respectively the total number of relocation operations and vehicles.
Equation (\ref{obj:maxDemands}) is the maximization of the total demand.
Equation (\ref{const:nbVROs}) expresses the bound on the total number of relocation.
Equation (\ref{const:nbVehicles}) expresses these on the total number of vehicles.
Equations (\ref{const:capacities}), (\ref{const:flowConservation}) and (\ref{const:flowDef}) are lastly flow constraints.
The total number of equations is around $2|\tegArcSet| + \nbStations \times \nbTimeSteps = \Theta(\nbStations^2 \cdot \nbTimeSteps)$.

\begin{flalign} \label{obj:maxDemands}
\max ~~\sum_{\substack{a\in \tegArcSet_2}} \varphi(a)
\end{flalign}
\begin{numcases}{s.t.}
\sum_{a \in \tegArcSet_3} \varphi(a) \leq \nbVROs & \label{const:nbVROs}\\
\sum_{a \in \tegArcSet} \varphi(a) \cdot \epsilon_a \leq \nbVehicles & \label{const:nbVehicles}\\
\varphi(a) \leq u(a) & $\forall a \in \tegArcSet$ \label{const:capacities}\\
\sum_{\substack{y\in \Gamma^-(x)}} \varphi((y,x)) = \sum_{\substack{y\in \Gamma^+(x)}} \varphi((x,y)) &$\forall x \in \tegNodeSet$ \label{const:flowConservation}\\
\varphi(a) \in \N & $\forall a\in\tegArcSet$ \label{const:flowDef}
\end{numcases}


\section{A polynomial sub-case}

Although the {\SDP} formulation sounds very close to the circulation problem definition, it presents some particularities that prevent us from its complexity evaluation.
As far as we know, the fact that the optimization is focused on a sub-set of arcs (carsharing demand arcs) and tries to find a feasible flow in the graph is not present in the literature neither as a standard problem nor related to some known one.
As a consequence, and despite our attempts, we do not succeeded in positioned this problem into a complexity class.
Probably, reduce or transform it into a classic flow problem may answer this question.

\medskip
Nevertheless, we obtained a complexity result for a specific instance of the problem. 
This section aims to prove that the determination of a flow satisfying all the carsharing demands without the constraints on the total number of relocation operations or vehicles is a polynomial problem.
Its formal definition, designated by {\SDPALLDEMANDS}, follows.

\bigskip
\noindent{\SDPALLDEMANDS}:
\begin{description}
\item[Inputs :] A Time Expanded Graph $\teg = (\tegNodeSet, \tegArcSet, u)$.
\item[Question :] Is there a feasible flow $\varphi$ in $\teg$ such that all the demands are fulfilled,\newline
\mbox{~~~~~~~~} \ie $\forall a \in \tegArcSet_2$, $\varphi(a) = u(a)$?
\end{description}

Let $I$ be an instance of \textsc{all-demands}.
We associate an instance of a max-flow problem $f(I)$ which network $\widehat{\teg}=(\widehat{\tegNodeSet}, \widehat{\tegArcSet}, \widehat{w})$ is defined as follows:
\begin{enumerate}
\item
Vertices are $\widehat{\tegNodeSet} = \tegNodeSet \cup \{s^\star, t^\star\} \cup \{s_a, t_a, a \in \tegArcSet_2\}$.
$s^\star$ and $t^\star$ are respectively the source and the sink of $\widehat{\teg}$, while $s_a$ and $t_a$ are two additional vertices associated to any demand arc $a \in \tegArcSet_2$.
\item
Arc set is $\widehat{\tegArcSet} = \tegArcSet_1 \cup \tegArcSet_3 \cup \{ (x,t_a), (t_a,t^\star), (s^\star, s_a), (s_a,y), \forall a=(x,y)\in \tegArcSet_2\}$.
\item
Maximum capacity of arcs are $\widehat{w}(a) = u(a)$ for $a \in \tegArcSet_1 \cup \tegArcSet_3$.
Otherwise, for any arc $a=(x,y) \in \tegArcSet_2$, $\widehat{w}((x,t_a))=\widehat{w}((t_a,t^\star))=\widehat{w}((s^\star, s_a))=\widehat{w}( (s_a,y))=u(a)$.
\end{enumerate}


\begin{figure}[t]
\centering
\begin{tikzpicture}[>=stealth, thick, scale=2]

\begin{scope} % LEFT SIDE
\node[station] (x) {$x$};
\node[station, below right of = x, node distance = 5cm] (y) {$y$};
\path[arcDemand] (x) edge[bend right, out=-15, in=160] node[swap, capacity]{$u(a)$} (y);
\end{scope}
%
\begin{scope}[shift={(3,0)}] % RIGHT SIDE
\node[station] (x) {$x$};
\node[station, below right of = x, node distance = 5cm] (y) {$y$};
\path[arcDemand, dashed] (x) edge[bend right, out=-15, in=160] (y);

\node[station, right of = x, node distance = 2.3cm] (ta) {$t_a$};
\node[station, left of = y, node distance = 2.3cm] (sa) {$s_a$};
\node[station, above left of = sa, node distance = 2.3cm] (ssource) {$s^\star$};
\node[station, below right of = ta, node distance = 2.3cm] (ssink) {$t^\star$};

\draw[->] (ssource) -> node[capacity, below left]{$u(a)$} (sa);
\draw[->] (sa)      -> node[capacity, below]{$u(a)$} (y);
\draw[->] (x)  -> node[capacity]{$u(a)$} (ta);
\draw[->] (ta) -> node[capacity]{$u(a)$} (ssink);

\node[left of = ssource, node distance = 2cm](dumb1){};
\node[right of = dumb1, node distance = 1cm](dumb2){};
\draw [->, >=stealth, line width = 4pt] (dumb1) -- (dumb2);
\end{scope}

\end{tikzpicture}
\caption{Transformation of an arc $a = (x,y) \in \tegArcSet_2$ into four arcs and two additional vertices $s_a$ and $t_a$  in $\widehat{\cal G}$. The corresponding maximum capacities are set to $u(a)$. Nodes $s^\star$ and $t^\star$ are unique and denote respectively the source and the sink nodes.}
\label{fig:tegTransformationPoly}
\end{figure}

Figure \ref{fig:tegTransformationPoly} illustrates the graph transformation and focuses on the adding graph components for a specific carsharing demand arc $a \in \tegArcSet_2$.
Note that this transformation is a polynomial function and does not depend on the structure of $\teg$.


\begin{theorem}\label{theo:pol}
Let an instance of \emph{\SDPALLDEMANDS} expressed by a TEG $\teg$.
There exists a feasible flow fulfilling all the demands of $\teg$ if and only if there exists a maximum flow in $\widehat{\teg}$ of value $\sum_{a\in \tegArcSet_2} u(a)$. 
\end{theorem}
\begin{proof}
Let suppose that $\varphi$ is a feasible flow of $\teg$ that fulfils all the demands, \ie for any arc $a \in \tegArcSet_2$, $\varphi(a) = u(a)$.
A flow $\widehat{\varphi}$ of $\widehat{\teg}$ may be built as follows:
\begin{enumerate}
\item
$\forall a \in \tegArcSet_1 \cup \tegArcSet_3$, 
$\widehat{\varphi}(a) = \varphi(a)$;
\item
For any arc $a=(x,y) \in \tegArcSet_2$, 
$\widehat{\varphi}((x,t_a))=\widehat{\varphi}((t_a,t^\star))=\widehat{\varphi}((s^\star, s_a))=\widehat{\varphi}((s_a,y))=u(a)$.
\end{enumerate}
We prove that $\widehat{\varphi}$ is a feasible flow of $\widehat{\teg}$ of value $\sum_{a \in \tegArcSet_2} u(a)$. 
Indeed, let consider a node $x \in \widehat{\tegNodeSet}$.
\begin{enumerate}
\item Let suppose first that $x \in \tegNodeSet$.
Then any demand arc ${a = (y,x) \in \tegArcSet_2}$ (\resp ${a = (x,y) \in \tegArcSet_2}$) of flow $\varphi(a)$ is associated in $\widehat{A}$ to an
arc $e = (s_a, x)$ (\resp $e = (x, t_a)$) with ${\widehat{\varphi}(e) = \varphi(a)}$.
Thus, 
\[\sum_{a \in \Gamma^-(\widehat{\teg}, x)} \widehat{\varphi}(a)=
\sum_{a \in \Gamma^-(\teg, x)} \varphi(a)=\sum_{a \in \Gamma^+(\teg, x)} \varphi(a)=
\sum_{a \in \Gamma^+(\widehat{\teg}, x)} \widehat{\varphi}(a)\text{.}\]

\item
For any arc $a=(z,y)\in \tegArcSet_2$, the two vertices $t_a$ and $s_a$ are such that
\[\sum_{e \in \Gamma^-(\widehat{\teg}, t_a)} \widehat{\varphi}(e)=\widehat{\varphi}((x, t_a))=\widehat{\varphi}((t_a, t^\star))=\sum_{e \in \Gamma^+(\widehat{\teg}, t_a)} \widehat{\varphi}(e)\]
and
\[\sum_{e \in \Gamma^-(\widehat{\teg}, s_a)} \widehat{\varphi}(e)=\widehat{\varphi}((s^\star,s_a))=\widehat{\varphi}((s_a, y))=\sum_{e \in \Gamma^+(\widehat{\teg}, s_a)} \widehat{\varphi}(e)\text{.}\]

\item
Lastly, \[\sum_{e \in \Gamma^-(\widehat{\teg}, t^\star)} \widehat{\varphi}(e)=\sum_{a\in \tegArcSet_2} u(a) \text{ ~~ and } \sum_{e \in \Gamma^+(\widehat{\teg}, s^\star)} \widehat{\varphi}(e)=\sum_{a\in \tegArcSet_2} u(a)\text{.}\]
\end{enumerate}
The consequence is that $\widehat\varphi$ is a feasible flow of $\widehat{\teg}$ of value $\sum_{a\in \tegArcSet_2} u(a)$. 

\medskip
Conversely, any feasible flow of $\widehat{\teg}$ of value $\sum_{a\in \tegArcSet_2} u(a)$ verifies
that, for any arc ${a=(x,y) \in \tegArcSet_2}$, 
$\widehat{\varphi}((x,t_a)) = \widehat{\varphi}((t_a,t^\star)) = \widehat{\varphi}((s^\star, s_a)) = \widehat{\varphi}( (s_a,y))=u(a)$.
A feasible flow for ${\teg}$ can be easily obtained by setting:
\begin{enumerate}
\item
$\forall a\in \tegArcSet_1 \cup \tegArcSet_3$, 
${\varphi}(a)=\widehat{\varphi}(a)$;

\item
For any arc $a=(x,y)\in \tegArcSet_2$, $\varphi(a)=u(a)$,
\end{enumerate}
the theorem.
\end{proof}

According to \cite{ahuja1993}, the existence of a maximum-flow of a fixed value is a polynomial problem.
The following corollary is thus a consequence of Theorem \ref{theo:pol}:
\begin{corollary}
\emph{\SDPALLDEMANDS} is polynomial.
\end{corollary}

\section{Conclusion}

\newpage
\addcontentsline{toc}{section}{Bibliography of chapter \thechapter}
\renewcommand{\bibname}{Bibliography of chapter \thechapter}
\putbib[bib/biblio]
\end{bibunit}
