\chapter{Introduction} \label{chap:introduction}
\begin{bibunit}[ieeetr]
\minitoc
\vspace{2cm}
%
\noindent
\begin{minipage}[c]{0.45\textwidth}
\includegraphics[width=\textwidth]{theBegining}
\end{minipage}
\hfill
\begin{minipage}[c]{0.45\textwidth}
\begin{abstract}
This chapter introduce the industrial background of the thesis.
It presents carsharing systems, their characteristics and specificities.
Finally, the thesis organization is exposed.
\end{abstract}
\end{minipage}

\newpage
\section{Context}
\subsection{Overview on today transportation systems}
Since the mid-twentieth century, the greater accessibility to the private car in industrialized countries has significantly improved the people mobility in urban areas. 
While this new mode of transportation greatly helped societies realize their aspiration for growth and prosperity, it also resulted in serious negative externalities.
For instance, pollution, excessive consumption of energy and time due to congestions problems, began to emerge.
To control, manage and deal with those problems, a lot of efforts are made to find alternative solutions \cite{mitchell_reinventing_2010}.\\

A citer et a inclure : \cite{le_vine_carsharing_2014}

\subsection{Challenges}

\section{Motivations}

\section{Problems addressed and thesis organization}

A recent study conducted by Correia and Antunes \cite{correia_optimization_2012} in 2012, inspired by \cite{fan_carsharing_2008}, proposed three
mathematical programming models to find the convenient choice of locations, number and size
of stations.
The objective was to maximize the operator profit taking into account all the revenues (price paid by clients) and costs involved (vehicle depreciation, vehicle maintenance
and parking maintenance).
Applied on the city of Lisbon, Portugal, the results showed that the model where the carsharing operator has full control over trip selection is the one which yield the maximum profit.
This was expected since it was the scheme offering the most freedom.
They also concluded that manage the imbalance situation (only at the end of each days in this study) is the key in a scenario where all demand should be satisfied, even if the client is charged a very high price.
Last, but not least, they also proved that the planning of stations location is intuitively dependent on the existence or non-existence of relocation operations which could
mitigate the effect of an uneven trip pattern and so allow the supply to expand.

\subsection{System dimensioning problem}

\subsection{Station location problem}

\subsection{The inclusion of energy components}


\newpage
\addcontentsline{toc}{section}{Bibliography of chapter \thechapter}
\renewcommand{\bibname}{Bibliography of chapter \thechapter}
\putbib[bib/biblio]
\end{bibunit}
