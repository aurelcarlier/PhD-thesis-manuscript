\chapter{Introduction} \label{chap:introduction}
\begin{bibunit}[ieeetr]
\minitoc
\vspace{2cm}
%
\noindent
\begin{minipage}[c]{0.45\textwidth}
\includegraphics[width=\textwidth]{theBegining}
\end{minipage}
\hfill
\begin{minipage}[c]{0.45\textwidth}
\begin{abstract}
%This chapter introduces the industrial background of the thesis.
%It presents carsharing systems, their characteristics and specificities.
%Finally, the thesis organization is exposed.
\end{abstract}
\end{minipage}

\newpage
\section{Context}
%\subsection{Overview on today transportation systems}
Since the mid-twentieth century, the greater accessibility to the private car in industrialized countries has significantly improved the people mobility.
In a lot of places, cars are still today the only transportation solution, especially in sparsely populated areas not or poorly served by public transport.
In addition, their ability to be used on-demand and perform door-to-door travels with a certain level of comfort makes them very attractive and not easily substituted by other alternative transportation modes.
The revolution that car has bought in terms of independence and transportation flexibility has far-reaching implications for the nature of societies \cite{jakle_lots_2004}.

\medskip
While this new mode of transportation greatly helped societies realize their aspiration for growth and prosperity, it also resulted in serious negative externalities.
Today, most of the major urban agglomeration suffer from congestion, lack of parking spaces, air and noise pollution.
\cite{maibach_handbook_2007} also identified several external costs (social, human, environmental, etc.) that can be attributed to car usage.
The list includes accidents, water and soil pollution, climate change and energy dependency.
As a consequence, worldwide public authorities placed the automotive in the heart of ecological concerns and over the last few years a lot of efforts are made to find alternative solutions \cite{mitchell_reinventing_2010}.
In the last decades, sharing has become a practical answer to many global socio-economic issues.
Although individual property is still predominant in our economic world, social organization has experienced that sharing information, goods or opinions can bring out sustainable answers.
Environmental consciousness for instance is consistent with the idea of sharing, not only because it is a new way of consumption, but also of its implications on objects (or services) design \cite{ciari_sharing_2012}.

\medskip
In the field of transportation, vehicle-sharing systems has emerged in the last twenty years as an innovative solution, dealing with both environmental and transportation issues.
Particularly implanted in dense urban areas, they are now recognized as an additional mobility supply and a transportation mode in itself.
Where traditional public transports, including buses, subways, trains and tramways fail at providing a great  flexibility in terms of covered area and time availability, vehicle-shared systems bring out significant advantages.
In 2007, the launch of V{\'e}lib' in Paris, France, disposed about $20,000$ self-service bikes over more than $1,000$ stations \cite{laporte_shared_2015}.
Its success has certainly asserted the idea and benefits of bicycle-sharing services around the world.
Today, more than $7,000$ systems involving over $800,000$ bicycles are currently operational.

Almost in the same time, numerous carsharing systems has been launched and their indisputable success, mainly in north America, Europe and Asia, suggests that they will continue to expand in the coming years.
Often praised and subsidized by public authorities (especially in Europe), carsharing provides the advantages of the automotive without the drawbacks of ownership.
Cars are available on-demand at defined locations, and rented for short periods of time.
Since each vehicle is more active on road and spend less time parked, carsharing participates to promote a more efficient and sustainable car utilisation.
Recent advances in real-time Information and Communication Technologies (ICT) also reinforced the service attractiveness, providing better accessibility to manage vehicle reservations \cite{jorge_carsharing_2013}.


In the early 1990s, due to increasing attention on environmental concerns, political authorities began to push for more fuel-efficient and lower-emissions vehicles.
As a response, and driven by promising advances in batteries and energy management, automakers initiated the development of electric vehicles.
Since 2008, electric cars are available to the public and as of 2015, the Renault-Nissan Alliance announced the milestone of $300,000$ units sold worldwide.
Include electric cars in their global fleet of vehicles came early in the history of carsharing.
As reliable experimental fields, carsharing systems with electric cars actually helped popularizing the concept of plug-in electric vehicles while reinforcing their positive impact on the environment.

\medskip
Fundamentally, carsharing systems embody a new paradigm with regard to car usage.
Indeed, the economical model depends on the time and the distance the user travelled.
Its operational principles modify drastically the way people perceive and utilize the vehicle.
The standard user behaviour evolves from a property-based usage (\ie using the vehicle because of owned it) to an utilization depending on real mobility needs.
This phenomenon partly explains why carsharing users drive less and for shorter periods of time, helping to enhance traffic conditions.


%A citer et a inclure : \cite{le_vine_carsharing_2014}\\
%Egallement : \cite{v-traffic_letat_2014}, \cite{la_mobilite_2008}, 

%\subsection{Challenges}
%\subsection{Carsharing}


\section{Motivations}

% \subsection{Industrial context}
All carmakers reflect on the evolution of their business and are considering to become active actors of the mobility. 
They do not want to be limited to the design, the sale, the manufacture or the repairs of vehicles.
The recent emergence of carsharing systems has raised a new potential business.
As a complement to the offer of public transport, deliver and operate carsharing fleets, is now a promising market.
To achieve these new objectives, car manufacturers have to elaborate methods and tools to assess the needs of a territory for this new mode of transport.
The methods and tools need to estimate demand, assist in the design of carsharing stations but also should provide economic arguments justifying the choice of this new mode of transport.
Moreover, primarily in large cities, the deployment of carsharing systems should move towards electric vehicles to take into account anti-pollution standards and even the prohibition of use of thermal motor vehicles.
The energy constraints, brought by electric cars must be considered by the methods and tools.
In addition, it will also be important to propose the operation methods of carsharing systems, taking into account their integration in the global system of multimodal transport.

% \subsection{Research environment}
% IRT SystemX

\section{Research scope and thesis organization}
Carsharing development is literally driven by the user flexibility.
The global trend to make the service more accessible greatly account for its success.
Historical systems managed by private operators are said \emph{station-based}.
The fleet of vehicles is spread out diverse locations providing parking places and optional and specific infrastructure as charging points for electric vehicles.
In this thesis, we focus on stations-based systems allowing \emph{one-way} trips where users can take-off and return a car in different locations.

\medskip
The scope of this thesis is dedicated to strategic and tactical problem arising in the design of those station-based one-way systems.
Addressed topic include optimal system dimensioning, optimal station locations and the consideration of energy components.
Some outputs of this work aim to evaluate the viability of a carsharing service with respect to a given territory and an estimation of a specific carsharing demand.
Decisions are made at a high level, and measured at both macro and micro scale.
The global purpose is to help decision makers to stand on the potential of such service before building and operating it.

\medskip
The next Chapter is dedicated to carsharing systems and their related problems.
A brief history, today's operating schemes and their benefits for the community and the environment are especially described.
Then, an overview of the literature looks over various topics related to carsharing.
More especially, demand modelling and strategic vehicles relocation aspects are unfolded as decisive challenges that design methods must encompass.
Finally, a random generator coping the lack of available carsharing data is described.
A synthesis and some outlooks are provided at the end of the chapter.


\medskip
Chapter \ref{chap:sdp} addresses the system dimensioning problem for one-way carsharing systems.
For fixed stations and demands, it consists in determine the relevant system components values (number of vehicles, parking places and relocation operations) allowing the system to run at its maximum efficiency.
A first sight at the existing literature about this topic is first given.
Then, we present a mathematical model based on time-expanded graphs and dealing with the system dimensioning optimization.
The proposed model accounts for vehicle relocation operations and maximizes the total number of demands.
Although its formulation is very closed to a maximum flow problem, the problem complexity is still unknown.
Nevertheless, a polynomial sub-case of the problem is exhibited in the end of the chapter.

\medskip
Chapter \ref{chap:sdpExp} is devoted to the evaluation of the system dimensioning MIP formulation.
Exact methods using both open-source an proprietary solvers are evaluated through four studies.
The first one computes optimal solutions maximizing the number of satisfied demands according to different values of the maximum number of vehicles and relocation operations during a day.
Computation times and a 3-pareto frontier are especially analysed.
Then, we evaluate the model scalability and challenge the exact method resolution on larger problem sizes.
A coupling between linear program building times and graph densities is detailed.
Also, important improvements applying another solver (CPLEX) are exposed.
Next studies focus on vehicle relocation strategies operating the system during a typical weekday.
More especially, strategies balancing the system at fixed times of the day are evaluated and compared to a baseline situation where vehicles can be relocated at any time.
The third study focuses on the impact on graph density whereas the last one examines improvements on computation times.


\medskip
Chapter \ref{chap:slp} deals with the station location problem and the electric vehicle inclusion in the optimization models.
An enhanced mathematical program is introduced and new decision variables cope with the selection of potential carsharing sites.
Additional constraints also accounts for a limited number of jockeys operating the vehicle relocations.
Besides, a dedicated reading of the problem highlights its strong dependence with previous statements and apprehends it as a generalized version of the system dimensioning model.
In the second part of the chapter, we introduce components related to electric cars.
More especially, the vehicle battery range and the power supply of charging points in stations are considered.
Then, we discuss the various issues that arise when considering a non-unit flows, especially the flow interpretation as individual vehicle routes.
We finally introduce the last MIP model including electric vehicles.
To overcome the combinatorial explosion of the number of variables, some model simplifications are then detailed.
In order to improve the performance of the solver, the description of a greedy heuristic terminates the chapter.
The algorithm helps the solver converging to an exact solution in finding rapidly a good lower integer bound.
A simple example illustrates the algorithm mechanics.


\medskip
Chapter \ref{chap:slpEnergyExp} is dedicated to the study of sufficient electric car batteries for the carsharing usage.
The optimal system containing $30$ electric vehicles is designed from data randomly generalised.
The results globally reflect realistic operative indicators similar to existing carsharing systems, such as the time spent on the road ($3$ hours) and the average number of demand each vehicle satisfies during the day ($5$ demands).
Moreover, specific outputs of the optimization allow to track every vehicle in the system as well as the specific amount of energy delivered in each station.
Besides, the observed maximum battery range (difference between the largest and the smallest level reach during the day) allows us to state on the minimum required battery capacity.
This result is especially useful for car makers wishing to dimension their vehicles for carsharing services.
Finally, we analyse the consequences of an increasing number of demands on the system sustainability and optimal battery range.


\medskip
Chapter \ref{chap:conclusion} ends this dissertation summarizing the results of the previous chapters.
Perspectives and future works are also suggested.


%\subsection{System dimensioning problem}
%\subsection{Station location problem}
%\subsection{The inclusion of energy components}


\newpage
\addcontentsline{toc}{section}{Bibliography of chapter \thechapter}
\renewcommand{\bibname}{Bibliography of chapter \thechapter}
\putbib[bib/biblio]
\end{bibunit}
