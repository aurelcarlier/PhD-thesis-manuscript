\chapter{SLP and energy experimentations} \label{chap:slpEnergyExp}
\begin{bibunit}[ieeetr]
\minitoc
\vspace{2cm}

\begin{minipage}[c]{.3\linewidth}
\includegraphics[width=\textwidth]{electricity}
\end{minipage}
\hfill
\begin{minipage}[c]{.7\linewidth}
\begin{abstract}
blabla\\
blabla\\
blabla\\
blabla\\
blabla\\
\end{abstract}
\end{minipage}


\section{Introduction}
Intro

\section{test graphs}

\begin{figure}[t]
\centering
\begin{tikzpicture}
\begin{axis}[ybar stacked, enlarge x limits=false, width=\linewidth, ybar, bar width=1pt] %,height=7cm ,ybar interval=0
\addplot table[col sep=semicolon, x index=0, y index=1]{tikz/D200.csv};
\addplot table[col sep=semicolon, x index=0, y index=2]{tikz/D200.csv};
\end{axis}
\end{tikzpicture}


%\begin{tikzpicture}
%\begin{axis}%[]
%
%\addplot table[x index=0, y index=1]{tikz/D.txt};
%
%width=.45\linewidth,
%xlabel=Number of stations,
%ylabel=Building times (in seconds),
%legend pos=north west,
%legend cell align=left,
%axis x line=bottom,
%axis y line=left,
%]
%
%\addplot[smooth,mark=*,color=green] plot coordinates {
%(10,211)
%(20,997)
%(30,1417)
%(40,2481)
%(50,4735)
%};
%\addlegendentry{$\nbTimeSteps = 288$}
%
%\addplot[smooth,mark=*,color=red] plot coordinates {
%(10,53)
%(20,234)
%(30,343)
%(40,632)
%(50,1437)
%};
%\addlegendentry{$\nbTimeSteps = 144$}
%
%\addplot[smooth,mark=*,color=blue] plot coordinates {
%(10,14)
%(20,59)
%(30,138)
%(40,147)
%(50,310)
%};
%\addlegendentry{$\nbTimeSteps = 72$}
%
%\end{axis}
%\end{tikzpicture}
\caption{A caption}
\label{plot:graphdensities}
\end{figure}


\section{Results}
The optimisation models and heuristic were run in an Intel® Xeon® processor 3.10 GHz, 16Gb RAM computer under Ubuntu using Cplex as a solver.
The code was written in java with the java API of Cplex. 

\subsection{Resolution time}
The first result we try to have is to know which instance we can resolve in a reasonable time.
The \textbf{Figure 6.3} gives us the resolution time for instances between 10 and 30 stations.
We can see that for 30 stations, some instances need 32 hours to be resolved.
Also, we can see that the heuristic have an impact for small instance but with 30 stations, it can have a impact null or even a bad impact. 

~\\~\\~\\Figure\\~\\~\\

We can also notice that the y axis is on a logarithmic scale which tell us that the time to solve an instance is exponential on the number of station.

\subsection{Energy versus no energy}
Our second experiment was to see the impact of the electric car opposed to fuel car in an optimal solution.
The \textbf{Figure 6.4} show us the difference in the number of request accepted, car used and vehicle relocation in each solution.
The number of request accepted stay the same with or without energy.
However, a solution with energy requires in average more cars  than a solution without.
If we restrain the number of car, then the energy solution cannot achieves the same number of request accepted than the solution without energy.
So, even if the solution seems to be equal in term of accepted request, the solution with  the energy constraint is different from a solution without such constraint.

\begin{figure}
\begin{tabularx}{\textwidth}{|X|c|c|c|c|}
\hline
Instances & Nb stations & Nb requests & Nb cars & Nb veh-reloc \\
\hline
Without energy & 10 & 74 & 12 & 20 \\
With energy & 10 & 74 & 14 & 19\\
\hline
With energy restraint \newline to 12 cars and 20 \newline vehicle relocations & 10 & 72 & 12 & 14\\
\hline  
Without energy & 20 & 361 & 60 & 50 \\
With energy & 20 & 361 & 60 & 50 \\
\hline
Without energy & 30 & 461 & 83 & 50\\
With energy & 30 & 461 & 84 & 50\\
\hline
Without energy & 50 & 790 & 149 & 50\\
With energy & 50 & 790 & 149 & 50\\
\hline
\end{tabularx}
\caption{Impacts of the energy on solutions}
\end{figure}

\subsection{Energy result}
The last experiment was to see how each car and station were used in a optimal solution with energy constraint.
We choose to create instances with the same ratio between station number, car and request as in the autolib system in Paris.
This mean for a system with $N$ stations, we have 3 x $N$ cars and 20 x $N$ requests.
The \textbf{Figure 6.5} shows the results for different instances.
We evaluate the average number of request, number of vehicle relocation, battery level and time spent in station of a car in an optimal solution.
The last column gives us the average energy a station delivers in a solution in comparison to maximum energy possible to delivers by a station.\\

\begin{figure}[h]
\newcolumntype{C}{>{\centering\arraybackslash}X}
\begin{tabularx}{\textwidth}{|C|C|C|C|C|C|}
\hline
Nb Stations & Nb requests done by a car & Nb veh-reloc by a car & Battery level ($\%$) & Time in station ($\%$) & Energy delivers by a station ($\%$)\\
\hline
10 & 6.06 & 0.96 & 35.21 & 73.65 & 37.48 \\
\hline
20 & 6.01 & 0.83 & 28.63 & 75.57 & 28.38 \\
\hline
30 & 5.48 & 0.76 & 33.39 & 71.43 & 27.42 \\
\hline
50 & 5.28 & 0.33 & 22.66 & 75.00 & 23.89 \\
\hline
\end{tabularx}
\caption{Results on average for a solution with energy}
\end{figure}

This result shows us that even if the energy has an impact in the solution as seen in \textbf{Figure 6.4}, the impact is small.
Indeed, a car stay in station almost $75\%$ of the day but has very low battery level.
And thanks to the last column, we can see that a car spends time in station without charging.
This show that in a system with the same proportion as autolib, the electric component isn't such a constraint as we might except. 
\newpage


\newpage
\addcontentsline{toc}{section}{Bibliography of chapter \thechapter}
\renewcommand{\bibname}{Bibliography of chapter \thechapter}
\putbib[bib/biblio]
\end{bibunit}
