\usepackage{lmodern}
\usepackage[frenchb,english]{babel} % --------------------- gestion des langues
\usepackage[utf8]{inputenc} % ----------------------------- gestion de l'encodage
\usepackage[T1]{fontenc} % -------------------------------- gestion de la fonte
\usepackage{eurosym}
\usepackage[dvips]{graphicx} % ---------------------------- https://www.ctan.org/pkg/graphicx ----------- insertion d'images
\usepackage{setspace}
\usepackage{wrapfig}
%\onehalfspacing

%%%%%%%%%%%%%%%%%%%%%%%%%%
% gestion des maths & algo
%%%%%%%%%%%%%%%%%%%%%%%%%%
\usepackage[fleqn]{amsmath}
\usepackage{amsfonts}
\usepackage{amssymb}
\usepackage[algoruled, linesnumbered]{algorithm2e} % ------------------------------ https://www.ctan.org/pkg/algorithm2e
\setlength{\interspacetitleruled}{8pt}
\usepackage[fleqn]{cases}
\usepackage{mathtools}
\DeclarePairedDelimiter\ceil{\lceil}{\rceil}
\DeclarePairedDelimiter\floor{\lfloor}{\rfloor}
\usepackage[autolanguage,np]{numprint}
\usepackage{icomma} % gestion des séparateurs de décimales: pas d'espace après une vigule

%\usepackage[authoryear]{natbib} % ---------------------------- https://www.ctan.org/pkg/natbib ------------- harmonisation des citations
\usepackage[]{chapterbib}
%\usepackage[globalcitecopy,sectionbib]{bibunits} % -------- https://www.ctan.org/pkg/bibunits sectionbib
\usepackage{url} % ---------------------------------------- https://www.ctan.org/pkg/url ---------------- insertion de liens url
\usepackage[pdftex,colorlinks=true,
                   urlcolor=black,
                   linkcolor=black,
                   citecolor=black]{hyperref} %------------ https://www.ctan.org/pkg/hyperref ----------- gestion des hyperref
\usepackage[justification=centering,margin=1cm]{caption}% - https://www.ctan.org/pkg/caption ------------ gestion des tables et des figures
\usepackage{geometry} % ----------------------------------- https://www.ctan.org/pkg/geometry ----------- gestion du layout des pages
\usepackage[svgnames]{xcolor}
\usepackage{subfig}
\usepackage{fancyhdr} % ----------------------------------- https://www.ctan.org/pkg/fancyhdr ----------- gestion des headers et footers
\usepackage{minitoc} % ------------------------------------ https://www.ctan.org/pkg/minitoc [english]
\usepackage{pdfpages} % ----------------------------------- https://www.ctan.org/pkg/pdfpages ----------- inclusion de pdf dans le doc

\usepackage{epigraph}
\usepackage{enumitem}
\setlist{itemsep=1.5mm} % définit l'espacement des listes 

%%%%%%%%%%%%%%%%%%%%%%
% gestion des tableaux
%%%%%%%%%%%%%%%%%%%%%%
\usepackage{tabularx}
\usepackage{multirow}
\usepackage{slashbox}

%%%%%%%%%%%%%%%%%%%
% tikz
%%%%%%%%%%%%%%%%%%%
\usepackage{tikz}
\usepackage{pgfplots}
\usepackage{rotating}

%\usepackage{multibib}
%\usepackage{layout}
%\usepackage{animate}
%\usepackage{multicol}
%\usepackage{array}
%\usepackage{lscape} % ------------------------------------- https://www.ctan.org/pkg/lscape

%\definecolor{bleu}{rgb}{0,0,0.7}
%\definecolor{vert}{rgb}{0,0.7,0}
%\definecolor{rouge}{rgb}{0.7,0,0}
%\definecolor{violet}{rgb}{0.5,0,0.5}
%\definecolor{correction_color}{rgb}{0.5,0.5,0}

%%%%%%%%%%%%%%%%%%%%%%%%%%%%%%%%%%%%%%%%%%%%%%%%%%%%%%%%%%%%%%%%%%%%%%%%%%%%%%%
% pour modifier la taille après les légendes
%\setlength{\belowcaptionskip}{2ex}
\setlength{\belowcaptionskip}{-6pt}
%%%%%%%%%%%%%%%%%%%%%%%%%%%%%%%%%%%%%%%%%%%%%%%%%%%%%%%%%%%%%%%%%%%%%%%%%%%%%%%%

%%%%%%%%%%%%%%%%%%%%%%%%%%%%%%%%%%%%%%%%%%%%%%%%%%%%%%%%%%%%%%%%%%%%%%%%%%%%%%%%
% pour mettre Figure et Table en gras dans les légendes
\makeatletter
\renewcommand{\fnum@figure}{\footnotesize\textbf{\figurename~\thefigure}}
\makeatother

\makeatletter
\renewcommand{\fnum@table}{\footnotesize\textbf{\tablename~\thetable}}
\makeatother
%%%%%%%%%%%%%%%%%%%%%%%%%%%%%%%%%%%%%%%%%%%%%%%%%%%%%%%%%%%%%%%%%%%%%%%%%%%%%%%%

%%%%%%%%%%%%%%%%%%%%%%%%%%%%%%%%%%%%%%%%%%%%%%%%%%%%%%%%%%%%%%%%%%%%%%%%%%%%%%%%
% Exemple de modifications du format de la page pour obtenir le format A4 propre
%%%%%%%%%%%%%%%%%%%%%%%%%%%%%%%%%%%%%%%%%%%%%%%%%%%%%%%%%%%%%%%%%%%%%%%%%%%%%%%%
\setlength{\hoffset}{0mm}
\setlength{\oddsidemargin}{0mm}
\setlength{\evensidemargin}{0mm}
\setlength{\voffset}{-0.4cm}
\setlength{\topmargin}{0mm}
% \setlength{\headheight}{0mm}
% \setlength{\headsep}{0mm}
\setlength{\textheight}{228mm}
\setlength{\textwidth}{159.2mm}
\setlength{\marginparwidth}{2cm}
\setlength{\footskip}{1.5cm}

%%%%%%%%%%%%%%%%%%%%%%%%%%%%%%%%%%%%%%%%%%%%%%%%%%%%%%%%%%%%%%%%%%%%%%%%%%%%%%%%%%%%
%% Exemple de definition des en-tete et des pieds de page
%%%%%%%%%%%%%%%%%%%%%%%%%%%%%%%%%%%%%%%%%%%%%%%%%%%%%%%%%%%%%%%%%%%%%%%%%%%%%%%%%%%%
%%%%%%%%%%%%%%%%%%%%%%%%%%%%%%%%%%%%%%%%%%%%%%%%%%%%%%%%%%%%%%%%%%%%%%%%%%%%%%%%%%%%
%% Pages quelconques
%%%%%%%%%%%%%%%%%%%%%%%%%%%%%%%%%%%%%%%%%%%%%%%%%%%%%%%%%%%%%%%%%%%%%%%%%%%%%%%%%%%%
\pagestyle{fancy} % définition du style fancy
\fancyhf{} % nettoyage des entete et pied de page precedents
\renewcommand{\chaptermark}[1]{\markboth{\bsc{\chaptername~\thechapter{} : }#1}{}}
\renewcommand{\sectionmark}[1]{\markright{#1}{}}
\fancyhf[HRE]{\bfseries \slshape \nouppercase{\leftmark}} % entete des pages paires
%\fancyhf[FL]{\bfseries \slshape \thepage} % pied de page des pages paires
\fancyhf[FR]{\bfseries \slshape \thepage} % pied de page des pages impaires
\fancyhf[HLO]{\bfseries \slshape \nouppercase{\rightmark}} % entete des pages impaires
%\fancyhf[FRO]{\bfseries \slshape \thepage} % pied de page des pages impaires
% redefinition de la taille des traits d'entete et de pied de page
\renewcommand{\headrulewidth}{0.5pt}
\renewcommand{\footrulewidth}{0.5pt}

%%%%%%%%%%%%%%%%%%%%%%%%%%%%%%%%%%%%%%%%%%%%%%%%%%%%%%%%%%%%%%%%%%%%%%%%%%%%%%%%%%%
% Première page de chapitre quasiment toujours dans le style plain
%%%%%%%%%%%%%%%%%%%%%%%%%%%%%%%%%%%%%%%%%%%%%%%%%%%%%%%%%%%%%%%%%%%%%%%%%%%%%%%%%%%
\fancypagestyle{plain}{ % définition du style plain
\fancyhf{} % nettoyage des entete et pied de page precedents
% pied de page des pages paires
\fancyhf[FL]{\bfseries \slshape Optimal design of a one-way carsharing system.}
\fancyhf[FR]{\bfseries \slshape   Aur{\'e}lien CARLIER}
\renewcommand{\headrulewidth}{0pt}
\renewcommand{\footrulewidth}{0.5pt}}
%%%%%%%%%%%%%%%%%%%%%%%%%%%%%%%%%%%%%%%%%%%%%%%%%%%%%%%%%%%%%%%%%%%%%%%%%%%%%%%%%%%


% définitions des répertoires figures
\graphicspath{
	{general-figs/}
	{chap-introduction/figs/}
	{chap-backgroundAndProblem/figs/}
	{chap-sdp/figs/}
	{chap-sdpExp/figs/}
	{chap-slpAndEnergy/figs/}	
	{chap-slpAndEnergyExp/figs/}
	{chap-conclusion/figs/}
	{tikz/}
}
%
%%%%%%%%%%%%%%%%%%%%%%%%%%%%%%%%%%%%%%%%%%%%%%%%%%%%%%%%%%%%%%%%%%%%%%%%%%%%%%%%%%%
% définitions de nouvelles commandes

% problems
\newcommand{\SDP}{[\textsc{sdp}]}
\newcommand{\SDPALLDEMANDS}{[\textsc{sdp-all-demands}]}
\newcommand{\SLP}{[\textsc{slp}]}
\newcommand{\ENERGY}{[\textsc{energy}]}

% classic math sets
\newcommand{\RR}{\mathbb{R}}
\newcommand{\Q}{\mathbb{Q}}
\newcommand{\Z}{\mathbb{Z}}
\newcommand{\N}{\mathbb{N}}

\newcommand{\timeStepSet}{\mathcal{T}}
\newcommand{\stationSet}{\mathcal{S}}
\newcommand{\sCapaSet}{\mathcal{Z}}
\newcommand{\demandSet}{\mathcal{D}}
\newcommand{\travelTimeSet}{\mathcal{H}}
\newcommand{\relocTimeStepSet}{\mathcal{R}}

\newcommand{\nbTimeSteps}{\mathsf{T}}
\newcommand{\nbStations}{\mathsf{S}}
\newcommand{\nbDemands}{\mathsf{D}}
\newcommand{\nbVROs}{R}
\newcommand{\nbVehicles}{V}
\newcommand{\nbMaxStations}{P}
\newcommand{\nbMaxJockeys}{J}

\newcommand{\teg}{\mathcal{G}}
\newcommand{\tegNodeSet}{\mathcal{X}}
\newcommand{\tegArcSet}{\mathcal{A}}
\newcommand{\tegCapacity}{u}
\newcommand{\enCons}{\gamma}


%\newcommand{\cX}{\mathcal{X}}
%\newcommand{\cA}{\mathcal{A}}

\newcommand{\Coupe}{\mathcal{C}}
\newcommand{\cCap}{u}

% are the following sets usefull ?
\newcommand{\cL}{\mathcal{L}}
%\newcommand{\cP}{\mathcal{P}}
\newcommand{\M}{\mathcal{M}}
\newcommand{\fC}{\mathfrak{C}}

% abréviations
\newcommand{\resp}{{\em resp. }}
\newcommand{\ie}{{\em i.e. }}
\newcommand{\eg}{{\em e.g. }}
\newcommand{\see}{{\em see. }}

% environnements
\newenvironment{proof}[1][Proof]{\noindent\textbf{#1.} }{\ \rule{0.5em}{0.5em}\bigskip}
\newenvironment{pbDefinition}[1][Definition]{\ \\ \noindent\textbf{[#1]}:}{}

\newtheorem{definition}{Definition}
\newtheorem{case}{Case}
\newtheorem{remark}{Remark}
\newtheorem{theorem}{Theorem}
\newtheorem{lemma}{Lemma}
\newtheorem{corollary}{Corollary}
\newtheorem{prop}{Property}
