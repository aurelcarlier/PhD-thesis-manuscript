% *-----------------------------------------------------------*
% *         Page de Garde (basé sur le modèle UPMC) 
% * Dernière modif le : 2012/02/13
% *-----------------------------------------------------------*
% * README: 
% *   - se compile avec pdflatex
% *   - l'entête de ce fichier sert pour générer une page
% *     unique. Ce texte latex peut s'inclure (partie comprise 
% *     entre les lignes et  ) dans un document existant.
% *-----------------------------------------------------------*

% - - - - - - - entête pour avoir une page unique 
\documentclass[11pt]{book}
\usepackage[english,french]{babel}
\usepackage[utf8]{inputenc}
\usepackage{graphicx}
\usepackage{tabularx}

\setlength{\textwidth}{16cm}
\setlength{\textheight}{25cm}
\setlength{\oddsidemargin}{-0cm}
\setlength{\topmargin}{-1cm}

\begin{document}

% - - - - - - - début de la page 
\thispagestyle{empty}

\includegraphics[width=.15\linewidth]{pics/logo_UPMC}

{\large

\vspace*{1cm}

\begin{center}

{\bf TH{\`E}SE DE DOCTORAT DE \\ l'UNIVERSIT{\'E} PIERRE ET MARIE CURIE}

\vspace*{0.5cm}

Sp{\'e}cialit{\'e} \\ [2ex]
{\bf Informatique}\ \\ 

\vspace*{0.5cm}

Ecole doctorale Informatique, T{\'e}l{\'e}communications et {\'e}lectronique (Paris)

\vspace*{1cm}

Pr{\'e}sent{\'e}e par \ \\

\vspace*{0.5cm}

{\Large {\bf Aur{\'e}lien CARLIER}}

\vspace*{1cm}
Pour obtenir le grade de \ \\[1ex]
{\bf DOCTEUR de l'UNIVERSIT{\'E} PIERRE ET MARIE CURIE} \ \\

\vspace*{1cm}

\end{center}

\flushleft{Sujet de la th{\`e}se :\ \\
\ \\
{\Large {\bf Titre\\ }}
  
\vspace*{1.5cm} 
\flushleft{soutenue le DATE}\\[2ex]
\flushleft{devant le jury compos{\'e} de :}\\[1ex]
\flushleft{\hspace{-1.3ex}\begin{tabularx}{\textwidth}{XlX}
Mme. Alix    {\sc Munier-Kordon} & Directrice de thèse  & Professeur at UPMC\\
Mme. Safia   {\sc Kedad Sidhoum} & Rapporteur           & Professeur at UPMC\\
M.   Roberto {\sc Wolfler Calvo} & Rapporteur           & Professeur at LIPN\\
M.   Witold  {\sc Klaudel}       & Examinateur          & \\
M.   Prénom  {\sc Nom}           & Examinateur          & \\
M.   Prénom  {\sc Nom}           & Examinateur          & \\
\end{tabularx}}

}
% - - - - - - - fin de la page
\end{document}
