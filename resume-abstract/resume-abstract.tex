%\chapter*{Résumé}
%Le résumé de la thèse en français


\chapter*{Abstract}
Carsharing is a mobility service allowing people to use and share a dedicated fleet of vehicles to satisfy transportation necessities.
As a excellent candidate to cope with environmental issues affecting nowadays several dense urban areas, it offers the flexibility of private vehicles without the drawbacks of car ownership.
In the field of mobility, the new economic model associated with sharing common goods has begun to reveal a new paradigm leading to numerous benefits.
Carsharing user behaviour evolves from a property-based application of the vehicles, to an usage aligned with real mobility needs.

\medskip
Many schemes are today implemented all around the world, and the evolution of carsharing  is promise to operate great expansion for the upcoming years.
The trend is toward more flexibility and more accessibility.
Although the round-trip model is still predominant, the implementation of one-way schemes, and more recently free-floating systems, have begun to emerged.
The carsharing one-way version allows the user to drop off the vehicles at any dedicated station of the system.
This gain of flexibility poses however several additional challenges, especially from a management and operational perspective.
The possibility to leave cars at any places of the system causes a resource imbalance that must be corrected to maintain a good level of service.

\medskip
Moreover, the recent introduction of electric cars in carsharing systems brings additional technical and practical constraints.
Indeed, shared vehicles spent more time on the roads and are more used than private ones.
The relatively limited autonomy of today's electric batteries may restrain the service in providing  reduced reachable range distances and committing the vehicles to recharge, making them unavailable.
Therefore, carsharing is a interesting and challenging topic, addressing various problems and dealing with strategic, tactical and operational aspect of those systems.

\medskip
This thesis focuses on the optimal design of one-way station-based carsharing systems.
We consider the system design through two structural aspects: the optimal system dimensioning (number of parking places, vehicles, battery capacities, etc.) and the identification of appropriate stations' locations.
Although the addressed problems do not directly concern the system management, some relevant aspects (like vehicle relocation operations) are nonetheless part of the models.
The modelling approach uses graph theory to represent the system dynamics over time and various optimization models (ILPs and MILPs) are proposed.
The objective is to deduce a optimal shape of the whole system (number of vehicles, parking places, jockeys, stations' locations, etc.) allowing to capture the maximum number of estimated time-dependant requests.
Mathematical programs consider integer flow variables accounting for vehicles moving dynamically in the system.
Electric vehicles are also included in an enhanced version and context related constraints ensure that vehicles cannot travel distances exceeding their battery range.
Power supply at charging points are also taken into account.
Then, the optimization allows to study the impact of different power supply technologies and settle the minimal autonomy a shared vehicle necessitate in this environment.

\medskip
All the models developed in this manuscript are applied to realistic case studies, using both random generated data and real estimated outputs of simulation tools.
To account for the necessary vehicle rebalancing, strategies including vehicle relocation operations managed by jockeys (employees of the carsharing operator) are considered.
We propose some graph simplifications reducing the problem size and leading to greatly improve solver capabilities as well as computation times.
A greedy heuristic helping to quickly find feasible solutions and initialize the solver is also proposed and illustrated.

\medskip
From an industrial perspective, several versions of the models have been implemented in a dedicated tool released as a Java Application Programming Interface (API).
The software, today  integrated into a dedicated IT platform devoted to the study of smart territories, provides ergonomic graphical interfaces and resolution methods, helping the actors of mobility design one-way carsharing systems with electric vehicles.
